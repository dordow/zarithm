\documentclass[12pt]{scrartcl}

\usepackage[T1]{fontenc}
\usepackage[ngerman,english]{babel}
\usepackage[backend=biber]{biblatex}
\addbibresource{/home/frank/docs/reference/literature.bib}
\AtEveryBibitem{\clearfield{url}}

\usepackage{graphicx}
\usepackage{url}
\usepackage{tikz}
\usepackage{units}
\usepackage{gensymb}
\usepackage{amssymb}
\usepackage{amsthm}
\usepackage{amsmath}
\usepackage{../source/zarithm}

\usepackage[mark]{gitinfo2}
\usepackage[outerbars]{changebar}
\usepackage[pdf,shell]{dottex}

%\usepackage{hyperref} % should be last but does not work then with zed-csp

%\usepackage{savesym}
\usepackage{zed-csp}
\usepackage{../source/zarithm}
% \savesymbol{because}
% %\usepackage[nothing]{todo}
% \usepackage{todo}
% \restoresymbol{TXF}{because}



%\nochangebars

\newcommand{\remark}[1]{ \vspace{1em}\textbf{\underline{#1}}\vspace{1em} }
%\newcommand{\comment}[1]{ \textbf{\underline{#1}} }

\newtheorem{prop}{Proposition}[section]
\newtheorem{lemma}{Lemma}[section]
\newtheorem{schemadef}{Schema}[section]
\newtheorem{genericdef}{Generic Definition}[section]
\newtheorem{zdef}{Definition}[section]

\begin{document}

% opening
\title{Polynomials}
\author{Frank Dordowsky}

% make the title area
\maketitle

\begin{abstract}
  This is a test paper for the Arithmetics in Z package and style,
  using polynomials.
\end{abstract}

%\tableofcontents
%\newpage

\section{Introduction}
\label{sec:intro}
The arithmetics of polynomials is used as a test bed for the
arithmetics in Z, type checked with \texttt{fuzz}.

As a prepararation, the generic finite distributed sum in
\texttt{zarithm.tex} is refined to the \emph{finite sum of elements of
Arithmos}:
\newcommand{\finsum}{\sum}
\begin{zdef}[Finite Sum]
  \label{zdef:finite-sum}
  
  \begin{axdef}
    \finsum : \nat \cross \nat \cross (\nat \fun \arithmos) \fun \arithmos \\
    \where
    \forall k, l : \nat; f: \nat \fun \arithmos | k \leq l @
    \finsum(k,l,f) = \findistsum((k \upto l) \dres f)
  \end{axdef}
\end{zdef}
%
A note on proofs: Proofs sometimes use the \texttt{argue} environment,
which is not type checked by \texttt{fuzz} according to the
\texttt{fuzz} manual \cite{Spivey2000}.
%
\section{Polynomials}
\label{sec:polynomials}
Now the central definition of the paper, the set of polynomials:
\begin{zdef}[Polynomials]
  \label{zdef:polynomials}
  A polynomial is actually a finite sequence of non-zero real numbers,
  which are members of a function from all natural numbers to the real
  numbers.
  \begin{zed}
    POLYNOMIAL == \{~p: \nat \fun \arithmos | (\exists N: \nat @ (\forall n:
    \nat | n > N @ p(n)=\azero))~\}
  \end{zed}
\end{zdef}

\begin{zdef}[Coeefficients of a Polynomial]
  \label{zdef:coefficient}
  The $i$-th coefficient of a polynomial is the real value at position
  $i$. 
  \begin{zed}
    coefficient == (\lambda p: POLYNOMIAL; i: \nat @ p(i))
  \end{zed}
\end{zdef}
%
With this definition it is
\begin{zed}
  coefficient \in POLYNOMIAL \cross \nat \fun \arithmos
\end{zed}
%
\newcommand{\zeropol}{\mathbf{0}}
\begin{zdef}[Zero Polynomial]
  \label{zdef:zero-polynomial}
  The \emph{zero polynomial} $\zeropol$ has only coefficients of value
  $\azero$:
  \begin{axdef}
    \zeropol : POLYNOMIAL
    \where
    \ran \zeropol = \{~\azero~\}
  \end{axdef}
\end{zdef}

\begin{zdef}[Degree of a Polynomial]
  The degree of a non-zero polynomial is the largest coefficient not equal to
  zero, and equal to $-1$ for the zero polynomial.
  \label{zdef:degree}
  \begin{axdef}
    deg : POLYNOMIAL \fun \nat
    \where
    deg~\zeropol = -1 \\
    \forall p: POLYNOMIAL | p \neq \zeropol @ deg~p = max~(\dom (p \nrres
  \{~\azero~\}))\\
  \end{axdef}
\end{zdef}

\begin{lemma}[Alternative Description of Degree]
  \label{lemma:alternative-degree}
  \begin{zed}
    \forall p: POLYNOMIAL; n: \nat @ deg~p = n \iff p(n) \neq \azero \land
    (\forall i : \nat | i > n @ p(i) = \azero)
  \end{zed}
\end{lemma}

\begin{proof}[Proof of lemma \ref{lemma:alternative-degree}]
  to  be completed
\end{proof}
%
Real and complex polynomials are those with coefficients in either
$\real$ or $\complex$ respectively:
\begin{zdef}[Complex Polynomials]
  \label{zdef:complex-polynomials}
  \emph{Complex polynomials} are polynomials where all coefficients
  are complex numbers:
  \begin{zed}
    ComplexPolynomial == \{~p: POLYNOMIAL | \ran~p \subseteq \complex~\}
  \end{zed}
\end{zdef}
%
The definition of real polynomials is based on that for complex
polynomials so that they form a subset relation.
\begin{zdef}[Real Polynomials]
  \label{zdef:real-polynomials}
  Real polynomials are polynomials where all coefficients are real numbers:
  \begin{zed}
    RealPolynomial == \{~p: ComplexPolynomial | \ran~p \subseteq \real~\}
  \end{zed}
\end{zdef}
%
With these definitions there is
\begin{zed}
  RealPolynomial \subseteq ComplexPolynomial \subseteq POLYNOMIAL
\end{zed}
%
\section{Addition and Multiplication of Polynomials}
\label{sec:addition-multiplication}
There are two operations on the set of polynomials, the addition and
the multiplication of polynomials.
\newcommand{\polyplus}{+}
\begin{zdef}[Addition of Polynomials]
  \label{zdef:polynomial-addition}
%%inop \polyplus 3 
  \begin{axdef}
    \_ \polyplus \_ : POLYNOMIAL \cross POLYNOMIAL \fun POLYNOMIAL\\
    \where
    \forall p, q : POLYNOMIAL @ p \polyplus q = (\lambda i: \nat @ p~i
    \aplus q~i)
  \end{axdef}
\end{zdef}

\begin{prop}[Zero Polynomial is neutral Element]
  \label{prop:zero-polynomial-neutral-element}
  The zero polynomial $\zeropol$ is the neutral element of the
  polynomial addition:
  \begin{zed}
    \forall p: POLYNOMIAL @ p \polyplus \zeropol = \zeropol \polyplus
    p = p
  \end{zed}
\end{prop}
% 
\newcommand{\polymult}{*}
\begin{zdef}[Multiplication of Polynomials]
  \label{zdef:polynomial-multiplication}
%%inop \polymult 4 
  \begin{axdef}
    \_ \polymult \_ : POLYNOMIAL \cross POLYNOMIAL \fun POLYNOMIAL\\
    \where
    \forall p, q : POLYNOMIAL @ p \polymult q = (\lambda i: \nat @
    \finsum(0,i,(\lambda j: \nat @ p(j) \amult q(i-j))))
  \end{axdef}
\end{zdef}
%
Todo: Polynomials form a commutative ring!!
% 
\begin{prop}[Additition and Multiplication of Real and Complex
  Polynomials are closed]
  \label{prop:closure-of-real-complex-addition-multiplication}
  Addition and Multiplication of real or complex polynomials result in
  real or complex polynomials respectively:
  \begin{zed}
    \ran ((RealPolynomial \cross RealPolynomial) \dres (\_ \polyplus
    \_ )) \subseteq RealPolynomial\\
    \ran ((ComplexPolynomial \cross ComplexPolynomial) \dres (\_ \polyplus
    \_ )) \subseteq ComplexPolynomial\\
    \ran ((RealPolynomial \cross RealPolynomial) \dres (\_ \polymult
    \_ )) \subseteq RealPolynomial\\
    \ran ((ComplexPolynomial \cross ComplexPolynomial) \dres (\_ \polymult
    \_ )) \subseteq ComplexPolynomial\\
  \end{zed}
\end{prop}
\begin{proof}[Proof of proposition
  \ref{prop:closure-of-real-complex-addition-multiplication}]
  The coefficients of the resulting polynomial of an addition or
  multiplication of two polynomials $p_1$ and $p_2$ are computed by addition and
  multiplication of the coefficients of $p_1$ and $p_2$. If all these
  coefficients are either real or complex, then are the sum or product
  of them, and so are the coefficients of the resulting polynomial.
\end{proof}
%
\begin{zdef}[Special binomial]
  \label{zdef:special-binnomial}
  $P$ is a constructor to produce the polynomial $x^n - r^n$ for a
  given $r \in \arithmos$ and $n \in \nat_1$.
  \begin{zed}
    P == (\lambda n: \nat_1; r: \arithmos @ (\zeropol \oplus \{~0 \mapsto
    r \apwr n~\}) \oplus \{~n \mapsto \aone ~\})
  \end{zed}
\end{zdef}

\begin{prop}[Factorization of special binomial]
  \label{prop:factorization-binomial}
  For a given $r \in \arithmos$ and $n \in \nat_1$ there is a polynomial
  $p$ with $(x^n-r^n) = p \polymult (x - r)$.
  \begin{zed}
    \forall n: \nat_1; r: \arithmos @ \exists p : POLYNOMIAL @ P(n,r) = p
    \polymult P(1,r)
  \end{zed}
\end{prop}

\begin{proof}[Proof of proposition \ref{prop:factorization-binomial}]
  The polynomial $p(x) = \sum_{i=0}^{n-1}r^{n-i-1}X^i$ does the
  trick. In Z:
  \begin{zed}
    \forall n: \nat_1; r: \arithmos @ \\
    \t1 (\LET p==\zeropol \oplus 
    (\lambda i : 0 \upto (n-1) @ r\apwr(n-i-1)) @ P(n,r) = p
    \polymult P(1,r))
  \end{zed}
  Formal proof to be completed.
\end{proof}

\section{Evaluation of Polynomials}
\label{sec:evaluation}

For the evaluation of polynomials we use the Horner schema. As a
preparation we define the left shifting of polynomials.
 
\begin{zdef}[Left-Shift of a Polynomial]
  \label{zdef:left-shift}
  \begin{axdef}
    lshift: POLYNOMIAL \fun POLYNOMIAL \\
    \where
    \forall p : POLYNOMIAL @ (\forall i: \nat @ (lshift~p)(i)=p(i+1))
  \end{axdef}
\end{zdef}

\begin{prop}[Left-shifting the zero Polynomial]
  \label{prop:left-shift-zero-polynomial}
  \begin{zed}
    lshift~\zeropol = \zeropol
  \end{zed}
\end{prop}

\begin{proof}[Proof of proposition
  \ref{prop:left-shift-zero-polynomial}]
  \begin{argue}
    \forall i : \nat @ (lshift~\zeropol)(i) = \zeropol(i+1) = \azero &
    def. of zero polynomial  \ref{zdef:zero-polynomial} and
    lshift \ref{zdef:left-shift}\\
    \vdash lshift~\zeropol = \zeropol \\
  \end{argue}
  
\end{proof}

\begin{lemma}[Degree of left-shifted Polynomial]
  \label{lemma:degree-left-shift}
  Left shifting a non-zero polynomial reduces its degree by one.
  \begin{zed}
    \forall p: POLYNOMIAL | p \neq \zeropol @ deg (lshift~p) = deg~p - 1
  \end{zed}
\end{lemma}

\begin{proof}[Proof of Lemma \ref{lemma:degree-left-shift}]
  It is necessary to distinguish two cases: $deg~p = 0$ and $deg~p >
  0$. Starting with  $deg~p = 0$:
  \begin{argue}
    p : POLYNOMIAL | deg~p = 0 @ p(0) \neq \azero \land
    (\forall i: \nat | i > 0 @ p(i) = \azero)\\
    \vdash (lshift~p) = \zeropol \\
    \vdash deg(lshift~p) = deg(\zeropol) = -1 = deg~p - 1
  \end{argue}
  Now for the case $deg~p > 0$:
  \begin{argue}
    p : POLYNOMIAL | deg~p > 0 @ p(deg~p) \neq \azero \land
    (\forall i: \nat | i > deg~p @ p(i) = \azero)\\
    \vdash (lshift~p)((deg~p)-1) = p(deg~p) \neq \azero \\
    \t1 \land (\forall i: \nat | i > ((deg~p)-1) @ (lshift~p)(i) =
    p(i+1) = \azero)\\
    \vdash deg(lshift~p) = deg~p - 1
  \end{argue}
\end{proof}

\begin{zdef}[Evaluation of Polynomials]
  \label{zdef:evaluation}
  The evaluation of a polynomial $p$ yields a function from $\arithmos$ to
  $\arithmos$:
  \begin{axdef}
    eval: POLYNOMIAL \fun (\arithmos \fun \arithmos)\\
    \where
    eval(\zeropol) = (\lambda v: \arithmos @ \azero)\\
    \forall p: POLYNOMIAL | p \neq \zeropol @ eval(p) = (\lambda v:
    \arithmos @ p(0) \aplus v \amult ((eval(lshift~p))(v)))
  \end{axdef}
\end{zdef}


\begin{prop}[Evaluation of Polynomials]
  \label{prop:evalutation}
  \begin{zed}
    \forall p: POLYNOMIAL; v: \arithmos | p \neq \zeropol @ (eval~p)(v) = \\
    \t1 \finsum(0,deg~p,(\lambda i: \nat @ (p~i)\amult(v \apwr i)))
  \end{zed}
 Note: in normal math notation this is simply
 \begin{equation*}
   \label{eq:math-notation-polynomial-evaluation}
   eval(p)(v) = \sum_{i=0}^n a_iv^i \text{ with } n = deg(p) \text{ and
   } a_i = p(i)
 \end{equation*}
\end{prop}
%
\begin{proof}[Proof of proposition \ref{prop:evalutation}]
  Induction over $deg~p$.
  \begin{argue}
    $deg~p = \azero$\quad \vdash (p~0 \neq \azero) \land (\forall i:
    \nat_1 @ p~i = \azero)\\
    \t1 \vdash lshift~p = \zeropol\\
    \t1 \vdash eval(lshift~p) = eval(\zeropol) = (\lambda v: \arithmos @
    \azero)\\
    \t1 \vdash eval(p) = (\lambda v: \arithmos @ p(0) \aplus v \amult
    \azero) = (\lambda v: \arithmos @ p(0))\\
    \t2 = \finsum(0,0,(\lambda i: \nat @ (p~i)\amult(v \apwr i)))
  \end{argue}

  We now assume that the proposition is correct for all polynomials of
  degree $n$ and consider polynomial $p$ of degree $n+1$.

  \begin{argue}
    \forall p: POLYNOMIAL | deg~p = n+1\\
    \vdash eval(p)(v) = p(0) \aplus v \amult (eval(lshift~p))(v) &
    Def. \ref{zdef:left-shift}\\
    \t1 = p(0) \aplus v \amult \finsum(0,deg(lshift~p),(\lambda i:
    \nat @ ((lshift~p)~i)\amult(v \apwr i))) & Lemma
    \ref{lemma:degree-left-shift} \\ & and induction\\
    \t1 = p(0) \aplus \finsum(0,deg(lshift~p),(\lambda i:
    \nat @ ((lshift~p)~i)\amult(v \apwr (i+1))))\\
    \t1 = p(0) \aplus \finsum(1,deg(lshift~p)+1,(\lambda i:
    \nat_1 @ ((lshift~p)~(i+1))\amult(v \apwr i)))\\
    \t1 = p(0) \aplus \finsum(1,deg~p,(\lambda i:
    \nat_1 @ p(i)\amult(v \apwr i)))\\
    \t1 = \finsum(0,deg~p,(\lambda i: \nat @ p(i)\amult(v \apwr i)))\\
  \end{argue}
\end{proof}
%
\begin{prop}[Evaluation of Real or Complex Polynomial yields Real or Complex Results]
  \label{prop:evaluation-real-complex-polynomial}
  The evaluation of a real value of a real polynomial yields a real
  result. The same applies to complex values and complex polynomials.
  \begin{zed}
%    \ran(\real \dres (RealPolynomial \dres eval)) \subseteq \real
    \forall p: RealPolynomial; r: \real @ (eval~p)(r) \in \real\\
    \forall p: ComplexPolynomial; c: \complex @ (eval~p)(c) \in \complex\\
  \end{zed}
\end{prop}
\begin{proof}[Proof of proposition \ref{prop:evaluation-real-complex-polynomial}]
  This is a corrolary of proposition \ref{prop:evalutation} since all
  terms are either real or complex.
\end{proof}

%
\section{Roots of a Polynomial}
\label{sec:roots}


\begin{zdef}[Roots of a Polynomial]
  \label{zdef:roots}
  \begin{axdef}
    roots : POLYNOMIAL \fun \power \arithmos\\
    \where
    roots~\zeropol = \arithmos\\
    \forall p : POLYNOMIAL | p \neq \zeropol @ roots~p = \{~r: \arithmos |
    (eval(p))(r) = \azero~\}
  \end{axdef}
\end{zdef}

\begin{prop}[Factor theorem]
  \label{prop:factor-theorem}
  The roots $r \in \arithmos$ of a polynomial $p$ can be factored out: $p
  = pf \polymult (x-r)$ with a polynomial $pf$.
  \begin{zed}
    \forall p: POLYNOMIAL; r: \arithmos | r \in roots~p @ \exists pf: POLYNOMIAL @ p =
    pf \polymult P(1,r)
  \end{zed}
\end{prop}
\begin{proof}[Proof of factor theorem \ref{prop:factor-theorem}]
  to be completed
\end{proof}
%
\begin{prop}[Fundamental Theorem of Algebra]
  Every non-constant complex polynomial has a root that is a complex
  number.
  \begin{zed}
    \forall p : ComplexPolynomial | deg~p > 0 @ roots~p \cap \complex
    \neq \emptyset
  \end{zed}
\end{prop}
%
% bibliography
\printbibliography{}

%\todos

\end{document}
%%% Local Variables:
%%% mode: latex
%%% TeX-master: t
%%% End:
