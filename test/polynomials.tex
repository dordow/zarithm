\documentclass[12pt]{scrartcl}

\usepackage[T1]{fontenc}
\usepackage[german,english]{babel}
\usepackage[backend=biber]{biblatex}
\addbibresource{/home/frank/docs/reference/literature.bib}
\AtEveryBibitem{\clearfield{url}}

\usepackage{graphicx}
\usepackage{url}
\usepackage{tikz}
\usepackage{units}
\usepackage{gensymb}
\usepackage{amssymb}
\usepackage{amsthm}
\usepackage{../source/zarithm}

\usepackage[mark]{gitinfo2}
\usepackage[outerbars]{changebar}
\usepackage[pdf,shell]{dottex}

%\usepackage{hyperref} % should be last but does not work then with zed-csp

%\usepackage{savesym}
\usepackage{zed-csp}
\usepackage{../source/zarithm}
% \savesymbol{because}
% %\usepackage[nothing]{todo}
% \usepackage{todo}
% \restoresymbol{TXF}{because}



%\nochangebars

\newcommand{\remark}[1]{ \vspace{1em}\textbf{\underline{#1}}\vspace{1em} }
%\newcommand{\comment}[1]{ \textbf{\underline{#1}} }

\newtheorem{prop}{Proposition}[section]
\newtheorem{schemadef}{Schema}[section]
\newtheorem{genericdef}{Generic Definition}[section]
\newtheorem{zdef}{Definition}[section]

\begin{document}

% opening
\title{Polynomials}
\author{Frank Dordowsky}

% make the title area
\maketitle

\begin{abstract}
  This is a test paper for the Arithmetics in Z package and style,
  using polynomials.
\end{abstract}

%\tableofcontents
%\newpage

% \section{Introduction}
% \label{sec:intro}

% Wieso weshalb warum


\begin{zdef}[Polynomials]
  \label{zdef:polynomials}
  \begin{zed}
    POLYNOMIAL == (\nat \ffun \real) \setminus \emptyset
  \end{zed}
\end{zdef}

\begin{zdef}[Coeefficients of a Polynom]
  \label{zdef:coefficient}
  The $i$-th coefficient of a polynom is the real value at position
  $i$. 
  \begin{zed}
    coefficient == (\lambda p: POLYNOMIAL; i: \nat | i \in \dom p @
    p(i))
  \end{zed}
\end{zdef}

\begin{zdef}[Degree of a Polynomial]
  The degree of a polynomial is the largest coefficient not equal to
  zero.
  \label{zdef:degree}
  \begin{zed}
    deg == (\lambda p: POLYNOMIAL @ max~(\dom (p \nrres
  \{~\azero~\})))
  \end{zed}
\end{zdef}
Attention: not correct, domain above may be emtpy, for emtpy sets max
not defined!

\begin{zdef}[Equivalence of Polynoms]
  \label{zdef:polynom-equivalence}
  Two polynomials are equivalent if they agree at all non-zero
  coefficients. 
\newcommand{\polyeq}{\equiv}
%%inrel \polyeq
  \begin{axdef}
    \_ \polyeq \_ : POLYNOMIAL \rel POLYNOMIAL\\
    \where \forall p, q : POLYNOMIAL @ (p \polyeq q) \iff ((p \nrres
    \{~\azero~\} )= (q \nrres \{~\azero~\}))
  \end{axdef}
\end{zdef}
It is clear from the definition that $\equiv$ is an equivalence
relation.

\begin{zdef}[Padding with Zeros]
  \label{zdef:padding-zeros}
  Padding means filling undefined coefficients with zeros:
  \begin{zed}
    pad == (\lambda p: POLYNOMIAL; N : \nat @ (\lambda i : 0 \upto N @
    \azero) \oplus p )
  \end{zed}
\end{zdef}

a padded polynomial is equivalent to the polynomial

addition and multiplication of polynomials

zero polynomial

evaluation of polynomials.

% 
% bibliography
\printbibliography{}

%\todos

\end{document}
%%% Local Variables:
%%% mode: latex
%%% TeX-master: t
%%% End:
