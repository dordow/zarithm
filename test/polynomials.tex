\documentclass[12pt]{scrartcl}

\usepackage[T1]{fontenc}
\usepackage[german,english]{babel}
\usepackage[backend=biber]{biblatex}
\addbibresource{/home/frank/docs/reference/literature.bib}
\AtEveryBibitem{\clearfield{url}}

\usepackage{graphicx}
\usepackage{url}
\usepackage{tikz}
\usepackage{units}
\usepackage{gensymb}
\usepackage{amssymb}
\usepackage{amsthm}
\usepackage{../source/zarithm}

\usepackage[mark]{gitinfo2}
\usepackage[outerbars]{changebar}
\usepackage[pdf,shell]{dottex}

%\usepackage{hyperref} % should be last but does not work then with zed-csp

%\usepackage{savesym}
\usepackage{zed-csp}
\usepackage{../source/zarithm}
% \savesymbol{because}
% %\usepackage[nothing]{todo}
% \usepackage{todo}
% \restoresymbol{TXF}{because}



%\nochangebars

\newcommand{\remark}[1]{ \vspace{1em}\textbf{\underline{#1}}\vspace{1em} }
%\newcommand{\comment}[1]{ \textbf{\underline{#1}} }

\newtheorem{prop}{Proposition}[section]
\newtheorem{schemadef}{Schema}[section]
\newtheorem{genericdef}{Generic Definition}[section]
\newtheorem{zdef}{Definition}[section]

\begin{document}

% opening
\title{Polynomials}
\author{Frank Dordowsky}

% make the title area
\maketitle

\begin{abstract}
  This is a test paper for the Arithmetics in Z package and style,
  using polynomials.
\end{abstract}

%\tableofcontents
%\newpage

% \section{Introduction}
% \label{sec:intro}

% Wieso weshalb warum


\begin{zdef}[Polynomials]
  \label{zdef:polynomials}
  \begin{zed}
    POLYNOMIAL == \{~p: \nat \fun \real | (\exists N: \nat @ (\forall n:
    \nat | n > N @ p(n)=\azero))~\}
  \end{zed}
\end{zdef}

\begin{zdef}[Coeefficients of a Polynom]
  \label{zdef:coefficient}
  The $i$-th coefficient of a polynom is the real value at position
  $i$. 
  \begin{zed}
    coefficient == (\lambda p: POLYNOMIAL; i: \nat @ p(i))
  \end{zed}
\end{zdef}

\newcommand{\zeropol}{\mathbf{0}}
\begin{zdef}[Zero Polynomial]
  \label{zdef:zero-polynomial}
  The \emph{zero polynomial} $\zeropol$ has only coefficients of value
  $\azero$:
  \begin{axdef}
    \zeropol : POLYNOMIAL
    \where
    \ran \zeropol = \{~\azero~\}
  \end{axdef}
\end{zdef}

\begin{zdef}[Degree of a Polynomial]
  The degree of a non-zero polynomial is the largest coefficient not equal to
  zero, and equal to $-1$ for the zero polynomial.
  \label{zdef:degree}
  \begin{axdef}
    deg : POLYNOMIAL \fun \nat
    \where
    deg~\zeropol = -1 \\
    \forall p: POLYNOMIAL | p \neq \zeropol @ deg~p = max~(\dom (p \nrres
  \{~\azero~\}))\\
  \end{axdef}
\end{zdef}

\newcommand{\polyplus}{+}
\begin{zdef}[Addition of Polynomials]
  \label{zdef:polynomial-addition}
%%inop \polyplus 3 
  \begin{axdef}
    \_ \polyplus \_ : POLYNOMIAL \cross POLYNOMIAL \fun POLYNOMIAL\\
    \where
    \forall p, q : POLYNOMIAL @ p \polyplus q = (\lambda i: \nat @ p~i
    \aplus q~i)
  \end{axdef}
\end{zdef}

\begin{prop}[Zero Polynomial is neutral Element]
  \label{prop:zero-polynomial-neutral-element}
  The zero polynomial $\zeropol$ is the neutral element of the
  polynomial addition:
  \begin{zed}
    \forall p: POLYNOMIAL @ p \polyplus \zeropol = \zeropol \polyplus
    p = p
  \end{zed}
\end{prop}

 multiplication of polynomials

evaluation of polynomials. maybe using Horner schema, left shift of
polynomials. 

% 
% bibliography
\printbibliography{}

%\todos

\end{document}
%%% Local Variables:
%%% mode: latex
%%% TeX-master: t
%%% End:
