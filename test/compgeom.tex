\documentclass[12pt]{scrartcl}

\usepackage[T1]{fontenc}
\usepackage[ngerman,english]{babel}
\usepackage[backend=biber]{biblatex}
\addbibresource{/home/frank/docs/reference/literature.bib}
\AtEveryBibitem{\clearfield{url}}

\usepackage{graphicx}
\usepackage{url}
\usepackage{tikz}
\usepackage{units}
\usepackage{gensymb}
\usepackage{amssymb}
\usepackage{amsthm}
\usepackage{amsmath}
\usepackage{../source/zarithm}

\usepackage[mark]{gitinfo2}
\usepackage[outerbars]{changebar}
\usepackage[pdf,shell]{dottex}

%\usepackage{hyperref} % should be last but does not work then with zed-csp

%\usepackage{savesym}
\usepackage{zed-csp}
\usepackage{../source/zarithm}
% \savesymbol{because}
% %\usepackage[nothing]{todo}
% \usepackage{todo}
% \restoresymbol{TXF}{because}

%\nochangebars

\newcommand{\remark}[1]{ \vspace{1em}\textbf{\underline{#1}}\vspace{1em} }
%\newcommand{\comment}[1]{ \textbf{\underline{#1}} }

\newtheorem{prop}{Proposition}[section]
\newtheorem{lemma}{Lemma}[section]
\newtheorem{schemadef}{Schema}[section]
\newtheorem{genericdef}{Generic Definition}[section]
\newtheorem{zdef}{Definition}[section]

\begin{document}

% opening
\title{Computational Geometry}
\author{Frank Dordowsky}

% make the title area
\maketitle

\begin{abstract}
  Jonthan Jacky has devoted a small chapter of his textbook
  \cite{Jacky1997a} on computational geometry using $\num$ for the
  coordinates. This is a test paper for the Arithmetics in Z package
  and style, translating the definitions there into real numbers.
\end{abstract}

%\tableofcontents
%\newpage

Chapter 19 of Jacky's textbook \cite{Jacky1997a} formalizes some basic
concepts of computational geometry with Z. The  coordinates are
defined using integer numbers. This paper simply translates the
definitions using the real numbers defined in the \texttt{zarithm.sty}
file and the \texttt{zarithm.tex} file.
%
% A note on proofs: Proofs sometimes use the \texttt{argue} environment,
% which is not type checked by \texttt{fuzz} according to the
% \texttt{fuzz} manual \cite{Spivey2000}.
%
\begin{zed}
  X == \real; Y == \real \\
  POINT == X \cross Y \\
  CONTOUR == \seq POINT \\
  SEGMENT == POINT \cross POINT \\
\end{zed}
%
These definitions are direct translations of the ones given in the
textbook. The definition of segments includes those that collapse to a
single point.
%
\begin{zed}
  px == first[X,Y]; py == second[X,Y], 
\end{zed}
%
Since $x$ and $y$ usually denote real variables, the projections were
renamed to $px$ and $py$.
%
\begin{zed}
  area2 == (\lambda a, b, c: POINT @ \\
  \t1 (px~a) \amult (py~b) \aminus (py~a)\amult (px~b) \aplus (py~a)\amult
  (px~c) \\
  \t1 \aminus (px~a) \amult (py~c) \aplus (px~b) \amult (py~c) \aminus
  (px~c) \amult (py~b))
\end{zed}
%
The function $area2$ calculates twice the area of a triangle given by
points $a,b,c$, but can be negative. It is
% 
\begin{zed}
  area2 \in POINT \cross POINT \cross POINT \fun \real
\end{zed}
%
Segments of a contour are calculated by:
%
\begin{zed}
  segments == (\lambda c : CONTOUR @ \{~a, b: POINT | \langle a,
  b\rangle \inseq c @ a \mapsto b~\} )
\end{zed}
%
It is
%
\begin{zed}
  segments \in CONTOUR \fun \power SEGMENT
\end{zed}
%
For a compact definition of polygones, two ``helper'' relations are
introduced: 
%%prerel between collinear
\begin{zed}
  ( between \_ ) == \{~x,y,z: \real | x \alt y \alt z \lor x \agt y
  \agt z~\}\\
  ( collinear \_ ) == \{~a, b, c: POINT | area2(a,b,c) = \azero~\}
\end{zed}
%
Three points are collinear if the triangle made up by them collapses
into a single line or even point, indicated by an area of zero.
%
\begin{axdef}
  on: POINT \rel SEGMENT \\
  \where
  \forall s: SEGMENT; a,c,d: POINT | s = (c,d) @ \\
  \t1 a \inrel{on} s \iff ( collinear(c,a,d) \\
  \t2 \land ((px~c \neq px~d \land between(px~c,px~a,px~d)) \\
  \t2 \lor (py~c \neq py~d \land between(py~c,py~a,py~d))))
\end{axdef}
%
If a point is on a segment it is never one of the corner points of the
segment, and these corner points must be different:
%
\begin{zed}
  \forall a,c,d: POINT; s: SEGMENT | s = (c,d) @ a \inrel{on} s
  \implies ( c \neq d  \land a \neq c \land a \neq d)
\end{zed}
%
\begin{axdef}
  touches: SEGMENT \rel SEGMENT \\
  \where
  \forall s_1, s_2 : SEGMENT; a,b,c,d: POINT | s_1 =(a,b) \land
  s_2=(c,d) @ \\
  \t1 s_1 \inrel{touches} s_2 \iff a \inrel{on} s_2 \lor b \inrel{on} s_2
\end{axdef}
%
If two segments are connected via their endpoints, as in a contour,
then they do \emph{not} touch, which is essential for the definition
of polygons.
% 
\begin{axdef}
  crosses: SEGMENT \rel SEGMENT \\
  \where
  \forall s_1, s_2 : SEGMENT; a,b,c,d: POINT | s_1 =(a,b) \land
  s_2=(c,d) @ \\
  \t1 s_2 \inrel{crosses} s_1 \iff area2(a,b,c) \amult area2(a,b,d) \alt \azero
\end{axdef}
%
\begin{axdef}
  intersects: SEGMENT \rel SEGMENT \\
  \where
  \forall s_1, s_2 : SEGMENT @ s_1 \inrel{intersects} s_2 \\
  \t1 \iff (s_1 \inrel{crosses} s_2 \land s_2 \inrel{crosses} s_1)
  \lor s_1 \inrel{touches} s_2 \lor s_2 \inrel{touches} s_1
\end{axdef}
%
Finally it is possible to define a polygon:
%
\begin{schema}{Polygon}
  c: CONTOUR
  \where
  \# c \geq 4 \\
  head~c = last~c\\
  front~c \in \iseq POINT \\
  \forall s_1, s_2 : segments~c | s_1 \neq s_2 @ \lnot (s_1 \inrel{intersects} s_2)
\end{schema}
% bibliography
\printbibliography{}

%\todos

\end{document}
%%% Local Variables:
%%% mode: latex
%%% TeX-master: t
%%% End:
