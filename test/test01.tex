\documentclass[12pt]{scrartcl}

\usepackage[T1]{fontenc}
\usepackage[german,english]{babel}
\usepackage[backend=biber]{biblatex}
\addbibresource{/home/frank/docs/reference/literature.bib}
\AtEveryBibitem{\clearfield{url}}

\usepackage{graphicx}
\usepackage{url}
\usepackage{tikz}
\usepackage{units}
\usepackage{gensymb}
\usepackage{amssymb}
\usepackage{amsthm}
\usepackage{../source/zarithm}

\usepackage[mark]{gitinfo2}
\usepackage[outerbars]{changebar}
\usepackage[pdf,shell]{dottex}

%\usepackage{hyperref} % should be last but does not work then with zed-csp

%\usepackage{savesym}
\usepackage{zed-csp}
% \savesymbol{because}
% %\usepackage[nothing]{todo}
% \usepackage{todo}
% \restoresymbol{TXF}{because}

%\newcommand{\Arithmos}{\mathbb{A}}
\newcommand{\Real}{\mathbb{R}}
\newcommand{\Rational}{\mathbb{Q}}

\newcommand{\azero}{0}
\newcommand{\aone}{1}
\newcommand{\aplus}{+}
\newcommand{\amult}{*}
\newcommand{\aminus}{-}
\newcommand{\adiv}{/}
\newcommand{\aneg}{-}
\newcommand{\ainv}{^{-1}}
\newcommand{\alt}{<}
\newcommand{\aleq}{\leq}
\newcommand{\agt}{>}
\newcommand{\ageq}{\geq}

%%% Local Variables:
%%% mode: latex
%%% TeX-master: t
%%% End:

%\newcommand{\dsum}{\mathrel{\Sigma}}


%\nochangebars

\newcommand{\remark}[1]{ \vspace{1em}\textbf{\underline{#1}}\vspace{1em} }
%\newcommand{\comment}[1]{ \textbf{\underline{#1}} }

\newtheorem{prop}{Proposition}[section]
\newtheorem{schemadef}{Schema}[section]
\newtheorem{genericdef}{Generic Definition}[section]
\newtheorem{zdef}{Definition}[section]

% Operator Definitions
\newcommand{\dsum}{\Sigma}

\begin{document}

% opening
\title{Rebalancing von Depots}
%\subtitle{Note 001}

\author{Frank Dordowsky}

% make the title area
\maketitle

\begin{abstract}
This is a template report useful if constructs of formal language Z is
used.
\end{abstract}

%\tableofcontents
%\newpage

\section{Einleitung}
\label{sec:einleitung}

Wieso weshalb warum

\section{Definitionen}
\label{sec:definitionen}
Dieser Abschnitt enthält generelle, vor allem mathematische,
Definitionen die im Folgenden verwendet werden.


\begin{genericdef}[Distributed Sum]
  \label{zdef:distributed-sum}
  For a finite list of real numbers, its sum is always defined and is
  recursively defined as:
  \begin{gendef}[I]
    \dsum : (I \ffun \Arithmos) \fun \Arithmos
    \where
    \dsum~\emptyset = \azero \\
    \forall f: I \ffun \Arithmos | f \neq \emptyset @ \\
    \t1 (\exists i: \dom f @ \dsum~f = (f~i) \aplus
    \dsum(\{~i~\} \ndres f))
  \end{gendef}
\end{genericdef}

\section{Wertpapiere}
\label{sec:wertpapiere}
Allgemeine Definitionen zu Wertpapieren.

\begin{zdef}[Wertpapier]
  \label{zdef:wertpapier}
  Wertpapiere sind alle an den Börsen gehandelten Papiere.
  \begin{zed}
    [Wertpapier]
  \end{zed}
\end{zdef}

Zu jedem Wertpapier gehört ein Kurs und ein Kurswert, der sich aus der
Anzahl der gehaltenen Anteile ergibt.

\begin{zdef}[Kurs und Kurswert]
  \label{zdef:kurs-kurswert}
  \begin{axdef}
    kurs : Wertpapier \fun \Rational\\
  \end{axdef}
\end{zdef}
kurs zu einem bestimmten Zeitpunkt, Einstandskurs, Kurswert,
Einstandswert





\section{Depot}
\label{sec:depot}
Anteil eines Assets, Wertpapiers am Portfolio


Für die Depotaufteilung werden Anlageklassen definiert:

\begin{zdef}[Anlageklassen]
  \label{zdef:anlageklassen}
  Die betrachteten Anlageklassen sind Aktien, Anleihen und Rohstoffe,
  gemäß Jacobs \cite{Jacobs2016}. Hierbei bedeuten $AE$ Aktien Europa,
  $AN$ Aktien Nordamerika, $AP$ Aktien Pazifikraum, $AS$ Aktien
  Schwellenländer, $RE$ Renten Euroraum, und $RO$ Rohstoffe. 
  \begin{zed}
    Anlageklasse ::= AE | AN | AP | AS | RE | RO 
  \end{zed}
\end{zdef}

\begin{zdef}[Aktien]
  \label{zdef:aktien}
  Die Aktien werden in einer Oberklasse zusammengefasst.
  \begin{axdef}
    Aktien : \power Anlageklasse
    \where
    Aktien = \{~AE, AN, AP, AS~\}
  \end{axdef}
\end{zdef}

\begin{zdef}[Depotaufteilung]
  \label{zdef:depotaufteilung}
  \begin{axdef}
    dpt\_auft : Anlageklasse \pfun \nat_1
    \where
    \dsum(dpt\_auft) = 100
  \end{axdef}
\end{zdef}
% \begin{genericdef}[ID]
%   \label{genericdef:id}
%   \begin{gendef}[X]
%     id: X \fun X \where id = \{~x: X @ x \mapsto x~\}
%   \end{gendef}
% \end{genericdef}

% \begin{zdef}[Time]
%   \label{zdef:time}
%   \begin{zed}
%     TIME == \nat \\
%   \end{zed}
% \end{zdef}


% \begin{zed}
%   TimeSteps \dres tmStp = id
% \also
%   \forall t: TIME @ tmStp(t) \leq t < tmStp(t) + 100
% \end{zed}


% \begin{prop}
%   \label{prop:some-proposition}
%   \begin{argue}
%     \vdash \forall ic: \num; sf: TimeSteps \fun VOLT; ts: TimeSteps | \\
%     \t1   ts \ne period @ \\
%     \t2 int(ic,sf)(ts+100) - int(ic,sf)(ts) = 100*ic*sf(ts)
%   \end{argue}
% \end{prop}

% \begin{proof}
%   The proof is as follows:

%   \begin{argue}
%     \forall ic: \num; sf: TimeSteps \fun VOLT; ts: TimeSteps | ts \ne
%     period \also ts + 100 \in TimeSteps \and ts + 100 \ne 0 \also
%     int(ic,sf)(ts+100) = int(ic,sf)(ts) + 100*ic*sf(ts)
%   \end{argue}
%   The last line follows from the definition of the integrator applied to
%   values of time steps, and the lemma follows from the last line above.
% \end{proof}

% We formalize the specification of the calibration algorithm (which is
% performed manually) with the following Z schema:

% \begin{schema}{Calibration}
%   Bcal?: BW \\
%   Tcal?: TEMP \\
%   mc?: ModuleCharacteristics
% \also
%   ro!: RampOffsets \\
%   rs!: RampSets \\
%   ic!: \num \\
%   fm!: FREQ
% \where
%   \forall TS: SensorTempSteps; ts: TimeSteps @\\ 
% \t1    rs!(Bcal?,TS,ts) = rs!(Bcal?,snsTempMin,ts)
% \also
%   \forall t: TIME @ \\
% \t1    abs(freqRampOf(ic!,ro!,rs!,mc?)(Bcal?,Tcal?,t) -\\
% \t2      idealRamp(Bcal?,fm!)(t)) \leq eps
% \end{schema}


% \begin{argue}
%   [DetermineIntegrationConstant] \vdash \forall ts: TimeStep | ts \ne period @ \\
% \t1   Ucal?(ts+100) - Ucal?(ts) = \\
% \t2     int(ic!,rampSlopeOf(rs?)(Bcal?,Tcal?))(ts+100) - \\
% \t3       int(ic!,rampSlopeOf(rs?)(Bcal?,Tcal?))(ts) = \\
% \t1     100*ic!*rampSlopeOf(rs?)(Bcal?,Tcal?)(ts) =    \\
% \t1   100*ic!*dac(rs?(Bcal?,snsTmpStp(Tcal?),ts))
% \end{argue}

% 
% bibliography
\printbibliography{}

%\todos

\end{document}
%%% Local Variables:
%%% mode: latex
%%% TeX-master: t
%%% End:
