\documentclass[12pt]{scrartcl}

\usepackage[T1]{fontenc}
\usepackage[english]{babel}
\usepackage[backend=biber]{biblatex}
\addbibresource{/home/frank/docs/reference/literature.bib}
\AtEveryBibitem{\clearfield{url}}

\usepackage{zarithm}
\usepackage{zed-csp} % must be included after zarithm
\usepackage[mark]{gitinfo2}
\usepackage[outerbars]{changebar}
\newcommand{\remark}[1]{ \vspace{1em}\textbf{\underline{#1}}\vspace{1em} }
%\newcommand{\comment}[1]{ \textbf{\underline{#1}} }

\begin{document}

%opening
\title{Arithmetic Definitions in  Z to be used with Fuzz}

\author{Frank Dordowsky}

% make the title area
\maketitle

\begin{abstract}
Goal is to create a prelude for fuzz that copes with arithmetics
according to \cite{Valentine2012} or \cite{Arthan1996}.
\end{abstract}

%\tableofcontents
%\newpage

\section{Introduction}
\label{sec:intro}

\section{Arithmos $\arithmos$}
\label{sec:arithmos}

$\arithmos$, which can be pronounced “arithmos”, is the encompassing
set of numbers whose whole extent is deliberately left
unspecified. This definition has been adopted from
\cite{Valentine2012}.

\begin{zed}
  [\arithmos] 
\end{zed}

%%inop \aplus 3
%%inop \aminus 3
\begin{axdef}
  \_ \aplus \_ : \arithmos \cross \arithmos \fun \arithmos\\
  \_ \aminus \_ : \arithmos \cross \arithmos \fun \arithmos\\
  \aneg : \arithmos \fun \arithmos\\
  \azero : \arithmos\\
  \aone : \arithmos\\
  \where
  \forall x : \arithmos @ x \aplus \azero = \azero \aplus x = x \\
  \forall x : \arithmos @ x \aplus (\aneg x) = \azero \\
  \forall x : \arithmos @ \aneg (\aneg x) = x \\
  \forall x,y: \arithmos @ x \aplus y = y \aplus x \\
  \forall x,y,z: \arithmos @ ( x \aplus y ) \aplus z = x \aplus (y
  \aplus z) \\
  \forall x,y : \arithmos @ x \aminus y = x \aplus (\aneg y)\\
\end{axdef}

\begin{zed}
  \arithmos_1 == \arithmos \setminus \{ \azero \}
\end{zed}

%%inop \amult 4
%%inop \adiv 4
%%postop \ainv
\begin{axdef}
  \_ \amult \_ : \arithmos \cross \arithmos \fun \arithmos\\
  \_ \adiv \_ : \arithmos \cross \arithmos_1 \fun \arithmos\\
  \_ \ainv : \arithmos_1 \fun \arithmos_1\\
  \where
  \forall x : \arithmos @ x \amult \aone = \aone \amult x = x \\
  \forall x : \arithmos_1 @ x \amult (x \ainv) = \aone \\
  \forall x : \arithmos_1 @ (x \ainv)\ainv = x \\
  \forall x,y: \arithmos @ x \amult y = y \amult x \\
  \forall x,y,z: \arithmos @ ( x \amult y ) \amult z = x \amult (y
  \amult z) \\
  \forall x: \arithmos; y : \arithmos_1 @ x \adiv y = x \amult (y
  \ainv)\\
  \forall x: \arithmos @ x \amult \azero = \azero \amult x =
  \azero\\
  \forall x, y, z: \arithmos @ x \amult ( y \aplus z ) = x \amult y
  \aplus x \amult z \\
\end{axdef}

\section{Real Numbers $\real$}
\label{sec:real-numbers}

\begin{axdef}
  \real : \power \arithmos\\
  \where
  % \azero \in \real\\
  % \aone \in \real\\
  \ran ( (\real \cross \real) \dres (\_ \aplus \_) ) \subset \real\\
  \ran ( (\real \cross \real) \dres (\_ \amult \_) ) \subset \real\\
  \ran (\real \dres (\aneg ))  \subset \real\\
  \ran (\real \dres (\_ \ainv ))  \subset \real\\
\end{axdef}

%%inrel \alt
\begin{axdef}
  \_ \alt \_ : \real \rel \real\\
  \where
  \forall x : \real @ \lnot x \alt x\\
  \forall x, y: \real @ x \alt y \lor x = y \lor y \alt x\\
  \forall x, y, z: \real | x \alt y \land y \alt z @ x \alt z \\
  \azero \alt \aone\\
\end{axdef}

%%inrel \aleq
%%inrel \agt
%%inrel \ageq
\begin{zed}
  (\_ \aleq \_) == (\_ \alt \_ ) \cup \id \real \\
  (\_ \agt \_) == (\_ \alt \_ ) \inv \\
  (\_ \ageq \_) == (\_ \agt \_ ) \cup \id \real \\
\end{zed}

\section{Rational Numbers $\rat$}
\label{sec:rational-numbers}

\begin{axdef}
  \rat : \power \real
  \where
  \azero \in \rat\\
  \aone \in \rat\\
  \ran ( (\rat \cross \rat) \dres (\_ \aplus \_) ) \subset \rat\\
  \ran ( (\rat \cross \rat) \dres (\_ \amult \_) ) \subset \rat\\
  \ran (\rat \dres (\aneg ))  \subset \rat\\
  \ran (\rat \dres (\_ \ainv ))  \subset \rat\\
\end{axdef}

\section{Integers and Rational Numbers}
\label{sec:integers-and-rationals}

\begin{axdef}
  numembd : \num \inj \rat \\
  \where
  numembd~0 = \azero \\
  numembd~1 = \aone \\
  \forall i, j : \num @ numembd (i+j) = numembd~i \aplus numembd~j \\
  \forall i, j : \num @ numembd (i*j) = numembd~i \amult numembd~j \\
  \forall i, j : \num | i <j @ numembd~i \alt numembd~j \\
\end{axdef}



% 
% bibliography
\printbibliography{}

%\todos

\end{document}
%%% Local Variables:
%%% mode: latex
%%% TeX-master: t
%%% End:
