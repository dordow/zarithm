\documentclass[12pt]{article}

\usepackage[T1]{fontenc}
\usepackage[english]{babel}

\usepackage{zed-csp} % must be included after zarithm
\usepackage{zarithm}
\usepackage[mark]{gitinfo2}
\newcommand{\remark}[1]{ \vspace{1em}\textbf{\underline{#1}}\vspace{1em} }
%\newcommand{\comment}[1]{ \textbf{\underline{#1}} }

\begin{document}

%opening
\title{Arithmetic Definitions in  Z to be used with Fuzz}

\author{Frank Dordowsky}

% make the title area
\maketitle

\begin{abstract}
  The intention of this \LaTeX\ file is to extend the mathematical
  tool-kit of the Z reference manual to include rational, real and
  complex numbers.
\end{abstract}

%\tableofcontents
%\newpage

\section{Introduction}
\label{sec:intro}
Rational and real numbers are often used in the specification of
embedded systems to represent environmental variables such as
distances, termperatures, voltage levels etc. The mathematical toolkit
of the Z reference manual (ZRM \cite{Spivey1998}) does not contain
rational and real numbers, whereas the number toolkit of the ISO Z
standard does. However, moving to standard (ISO) Z \cite{ISO13568} is
not always the favourite option, especially if a large amount of Z
specifications in ZRM style is available. Moreover, most Z textbooks
have been written prior to standard Z. For those who wish to stick
with ZRM, this set of two files (\texttt{zarithm.sty} and
\texttt{zarithm.tex}, this file) are intended to provide a solution.

The file \texttt{zarithm.sty} is only used for typesetting, whereas
\texttt{zarithm.tex} includes the Z definitions of the types for
rational, real and complex numbers, their basic operations and
numerical comparisons. The Z definitions are sufficient for type
setting, not for automated proof. If the user intends to use automated
proof, there is probably no better way than to transit to standard Z.

The ZRM comes along with the tool \texttt{fuzz}, a typechecker. The
approach described here makes use of the fact that fuzz accepts more
than one \LaTeX file and remembers the Z definitions of the files
provided first.

The document is structured as follows: the next section
\ref{sec:arithmos} defines an artifical set of numbers named
Arithmos. Section \ref{sec:real-numbers} introduces real numbers,
section \ref{sec:rational-numbers} rational numbers, and section
\ref{sec:complex-numbers} complex numbers. Section
\ref{sec:integers-and-rationals} discusses the relation between
integers and rational numbers. Section \ref{sec:additional-functions}
introduces a couple of functions that are often used, and section
\ref{sec:latex-commands} lists the \LaTeX{} commands that are used for
type setting number sets, operations and relations.
%
\section{Arithmos $\arithmos$}
\label{sec:arithmos}

$\arithmos$, which can be pronounced “arithmos”, is the encompassing
set of numbers whose whole extent is deliberately left
unspecified. This definition has been adopted from Valentine
\cite{Valentine2012}.

\begin{zed}
  [\arithmos] 
\end{zed}

$\arithmos$ is a field with respect to two operations $\aplus$ and
$\amult$, and neutral elements $\azero$ and $\aone$. Note that these
are not identical to the elements $0$ and $1$ defined in $\num$.

The following axiomatic definition introduces addition, subtraction,
and negation, together with some important constants.
%
%%inop \aplus 3
%%inop \aminus 3
\begin{axdef}
  \_ \aplus \_ : \arithmos \cross \arithmos \fun \arithmos\\
  \_ \aminus \_ : \arithmos \cross \arithmos \fun \arithmos\\
  \aneg : \arithmos \fun \arithmos\\
  \azero : \arithmos\\
  \aone : \arithmos\\
  \atwo : \arithmos\\
  \aten : \arithmos\\
  \where
  \forall x : \arithmos @ x \aplus \azero = \azero \aplus x = x \\
  \forall x : \arithmos @ x \aplus (\aneg x) = \azero \\
  \forall x : \arithmos @ \aneg (\aneg x) = x \\
  \forall x,y: \arithmos @ x \aplus y = y \aplus x \\
  \forall x,y,z: \arithmos @ ( x \aplus y ) \aplus z = x \aplus (y
  \aplus z) \\
  \forall x,y : \arithmos @ x \aminus y = x \aplus (\aneg y)\\
  \atwo = \aone \aplus \aone \\
  \aten = \atwo \aplus \atwo \aplus \atwo \aplus \atwo \aplus \atwo \\
\end{axdef}
% 
$\arithmos_1$ is the set of non-zero numbers:
%
\begin{zed}
  \arithmos_1 == \arithmos \setminus \{ \azero \}
\end{zed}
%
The next axiomatic definition introduces multiplication, division and
inversion. 
%
%%inop \amult 4
%%inop \adiv 4
%%postop \ainv
\begin{axdef}
  \_ \amult \_ : \arithmos \cross \arithmos \fun \arithmos\\
  \_ \adiv \_ : \arithmos \cross \arithmos_1 \fun \arithmos\\
  \_ \ainv : \arithmos_1 \fun \arithmos_1\\
  \where
  \forall x : \arithmos @ x \amult \aone = \aone \amult x = x \\
  \forall x : \arithmos_1 @ x \amult (x \ainv) = \aone \\
  \forall x : \arithmos_1 @ (x \ainv)\ainv = x \\
%  \forall x,y: \arithmos @ x \amult y = y \amult x \\
  \forall x,y,z: \arithmos @ ( x \amult y ) \amult z = x \amult (y
  \amult z) \\
  \forall x: \arithmos; y : \arithmos_1 @ x \adiv y = x \amult (y
  \ainv)\\
  \forall x: \arithmos @ x \amult \azero = \azero \amult x =
  \azero\\
  \forall x, y, z: \arithmos @ x \amult ( y \aplus z ) = x \amult y
  \aplus x \amult z \\
\end{axdef}
%
Note that the multiplication is not communtative since it is intended
to include Hamilton quaternions in $\arithmos$ which are not
commutative with respect to multiplication. 

An important function that is defined for Arithmos is the \emph{power
  function}:
%%inop \apwr 4
\begin{axdef}
  \_ \apwr \_ : \arithmos \cross \num \pfun \arithmos \\
  \where
  \forall n: \nat @ \azero \apwr n = \azero\\
  \forall x: \arithmos @ x \apwr 0 = \aone\\
  \forall x: \arithmos_1; z : \num @ x \apwr (z+1) = x \amult (x \apwr z)\\
\end{axdef}

Another function to be used in the following is the \emph{square root}
which may not be defined for all elements of $\arithmos$. It is
therefore defined as a partial function on $\arithmos$:
%
\begin{axdef}
  sqrt : \arithmos \pfun \arithmos
  \where
  \forall x : \dom sqrt @ sqrt(x) \amult sqrt(x) = x
\end{axdef}
%
The square root function has the following properties, derived from
its definition:
\begin{zed}
  sqrt(\azero) = \azero\\
  sqrt(\aone) = \aone\\
\end{zed}
%
\section{Real Numbers $\real$}
\label{sec:real-numbers}
The set of real numbers $\real$ is a subset of $\arithmos$, which
contains the constants defined in section \ref{sec:arithmos}.
%
\begin{axdef}
  \real : \power \arithmos\\
  \where
  \azero \in \real\\
  \aone \in \real\\
  \atwo \in \real\\
  \aten \in \real\\
  \ran ( (\real \cross \real) \dres (\_ \aplus \_) ) \subset \real\\
  \ran ( (\real \cross \real) \dres (\_ \amult \_) ) \subset \real\\
  \ran (\real \dres (\aneg ))  \subset \real\\
  \ran (\real \dres (\_ \ainv ))  \subset \real\\
  \forall x,y: \real @ x \amult y = y \amult x \\
\end{axdef}

The real numbers are closed with respect to addition, multiplication,
negation and inversion. Since $\arithmos$ is not commutative with
respect to multiplication, it must be introduced here for the real
numbers. 

For real numbers, there is a relation ``$<$'':
%
%%inrel \alt
\begin{axdef}
  \_ \alt \_ : \real \rel \real\\
  \where
  \forall x : \real @ \lnot x \alt x\\
  \forall x, y: \real @ x \alt y \lor x = y \lor y \alt x\\
  \forall x, y, z: \real | x \alt y \land y \alt z @ x \alt z \\
  \azero \alt \aone\\
\end{axdef}
%
The definition of the other relations $\aleq$, $\agt$ and $\ageq$ can
be traced back to this relation:
%
%%inrel \aleq
%%inrel \agt
%%inrel \ageq
\begin{zed}
  (\_ \aleq \_) == (\_ \alt \_ ) \cup \id \real \\
  (\_ \agt \_) == (\_ \alt \_ ) \inv \\
  (\_ \ageq \_) == (\_ \agt \_ ) \cup \id \real \\
\end{zed}
%
There are shortcuts for positive and negative real numbers:
%
\begin{zed}
  \realplus == \{~r :\real | r \agt \azero~\}\\
  \realminus == \real \setminus ( \realplus \cup \{~\azero~\})\\
  \real_1 == \realplus \cup \realminus \\
\end{zed}
%
% Some standard vector XXX over the real numbers are defined as follows:
% \begin{zed}
%   \realtwod == \real \cross \real \\
%   \realthreed == \realtwod \cross \real \\
%   \realfourd == \realthreed \cross \real \\
% \end{zed}
%
% Attempt:
% \begin{zed}
%   pr2d_1 == 
% \end{zed}

For the square root function restricted to the real numbers there are
some additional properties:
%
\begin{zed}
  \realplus \cup \{~\azero~\} \subseteq \dom sqrt\\
  \dom (sqrt \rres \real) = \realplus \cup \{~\azero~\}\\
  \ran (\real \dres sqrt) = \realplus \cup \{~\azero~\}\\
\end{zed}
%
\section{Rational Numbers $\rat$}
\label{sec:rational-numbers}
The set of rational numbers $\rat$ is a subset $\real$.
%
\begin{axdef}
  \rat : \power \real
  \where
  \azero \in \rat\\
  \aone \in \rat\\
  \atwo \in \rat\\
  \aten \in \rat\\
  \ran ( (\rat \cross \rat) \dres (\_ \aplus \_) ) \subset \rat\\
  \ran ( (\rat \cross \rat) \dres (\_ \amult \_) ) \subset \rat\\
  \ran (\rat \dres (\aneg ))  \subset \rat\\
  \ran (\rat \dres (\_ \ainv ))  \subset \rat\\
\end{axdef}
The constants defined in section \ref{sec:arithmos} are also members
of $\rat$.

There are shortcuts for the positive, negative and non-zero rational numbers:
\begin{zed}
  \ratplus == \{~q :\rat | q \agt \azero~\}\\
  \ratminus == \rat \setminus ( \ratplus \cup \{~\azero~\}) \\
  \rat_1 == \ratplus \cup \ratminus \\
\end{zed}
%
\section{Quaternions}
\label{sec:quaternions}
The set of \emph{Quaternions} $\quaternions$ is also considered as a
subset of $\arithmos$, which contains the real numbers and the
rational numbers.

Quaternions are defined via three (actually four) basis quaternions:
%
\begin{axdef}
  \iu, \ju, \ku : \arithmos
  \where
  \iu \amult \iu = \ju \amult \ju = \ku \amult \ku = \aneg \aone\\
  \iu \amult \ju = \aneg \ju \amult \iu = \ku \\
  \ju \amult \ku = \aneg \ku \amult \ju = \iu \\
  \ku \amult \iu = \aneg \iu \amult \ku = \ju
\end{axdef}

The fourth basis quaternion is $\aone$.

The subset $\quaternions \subset \arithmos$ is defined as follows:
%
\begin{axdef}
  \quaternions : \power \arithmos
  \where 
  \quaternions = \{~q_0, q_1, q_2, q_3 : \real @ q_0 \amult \aone
  \aplus q_1 \amult \iu \aplus q_2 \amult \ju \aplus q_3 \amult \ku~\}
\end{axdef}
%
Since $\aone, \iu, \ju, \ku$ form a basis, the coefficients $q_i$ are
unique.
%This allows the definition of \emph{projection functions}:
%
% \begin{axdef}
%   Hpr\_1, Hpr\_i, Hpr\_j, Hpr\_k : \quaternions \fun \real
%   \where
%   \forall q: \quaternions @ q = Hpr\_1~q \amult \aone \aplus Hpr\_i~q \amult
%   \iu \aplus Hpr\_j~q \amult \ju \aplus Hpr\_k~q \amult \ku
% \end{axdef}
%
% From previous definitions the addition of two quaternions can be
% calculated:
% \begin{zed}
%   \forall q1, q2 : \quaternions @ \\
%   \t1 (\LET q1_0 == Hpr\_1~q1 ;\\
%   \t2 q1_1  == Hpr\_i~q1 ;\\
%   \t2 q1_2  == Hpr\_j~q1 ;\\
%   \t2 q1_3  == Hpr\_k~q1 ;\\
%   \t2 q2_0 == Hpr\_1~q2 ;\\
%   \t2 q2_1  == Hpr\_i~q2 ;\\
%   \t2 q2_2  == Hpr\_j~q2 ;\\
%   \t2 q2_3  == Hpr\_k ~q2 @ \\
%   \t1 q1 \aplus q2 = \\
%   \t2 ( q1_0 \aplus q2_0 ) \amult \aone \aplus ( q1_1 \aplus q2_1 ) \amult
%   \iu \aplus ( q1_2 \aplus q2_2 ) \amult \ju \aplus ( q1_3 \aplus q2_3
%   ) \amult \ku ) \\
% \end{zed}
%
Every quaternion can be expressed as a unique quadruple of real
coefficients which is used below to express addition and
multiplication in terms of these coefficients, using the equations for
$\aone, \iu, \ju, \ku$ above. For the addition of
quaternions there is:
%
\begin{zed}
  \forall q1, q2 : \quaternions @ \\
  \t1 \exists_1 q1_0, q1_1, q1_2, q1_3, q2_0, q2_1, q2_2, q2_3 :
  \real @ \\
  \t2 (q1 = q1_0 \amult \aone  \aplus q1_1 \amult \iu \aplus q1_2
  \amult \ju \aplus q1_3 \amult \ku ) \land \\
  \t2 (q2 = q2_0 \amult \aone  \aplus q2_1 \amult \iu \aplus q2_2
  \amult \ju \aplus q2_3 \amult \ku ) \land \\
  \t1 (q1 \aplus q2 = \\
  \t2 ( q1_0 \aplus q2_0 ) \amult \aone \aplus ( q1_1 \aplus q2_1 ) \amult
  \iu \aplus ( q1_2 \aplus q2_2 ) \amult \ju \aplus ( q1_3 \aplus q2_3 
  )  \amult \ku ) \land \\
  \t1 (q1 \aminus q2 = \\
  \t2 ( q1_0 \aminus q2_0 ) \amult \aone \aplus ( q1_1 \aminus q2_1 ) \amult
  \iu \aplus ( q1_2 \aminus q2_2 ) \amult \ju \aplus ( q1_3 \aminus q2_3 
  )  \amult \ku )
\end{zed}
%
For multiplication of quaternions:
\begin{zed}
  \forall q1, q2 : \quaternions @ \\
  \t1 \exists_1 q1_0, q1_1, q1_2, q1_3, q2_0, q2_1, q2_2, q2_3 :
  \real @ \\
  \t2 (q1 = q1_0 \amult \aone  \aplus q1_1 \amult \iu \aplus q1_2
  \amult \ju \aplus q1_3 \amult \ku ) \land \\
  \t2 (q2 = q2_0 \amult \aone  \aplus q2_1 \amult \iu \aplus q2_2
  \amult \ju \aplus q2_3 \amult \ku ) \land \\
  \t1 q1 \amult q2 =\\
  \t2 ((q1_0 \amult q2_0 \aminus q1_1 \amult q2_1 \aminus q1_2 \amult q2_2
  \aminus q1_3 \amult q2_3 ) \amult \aone \aplus \\
  \t2 (q1_0 \amult q2_1 \aplus q2_0 \amult q1_1 \aplus q1_2 \amult
  q2_3 \aminus q2_2 \amult q1_3 ) \amult \iu \aplus \\
  \t2 (q1_0 \amult q2_2 \aplus q2_0 \amult q1_2 \aplus q1_3 \amult
  q2_1 \aminus q2_3 \amult q1_1 ) \amult \ju \aplus \\
  \t2 (q1_0 \amult q2_3 \aplus q2_0 \amult q1_3 \aplus q1_1 \amult
q2_2 \aminus q2_1 \amult q1_2 ) \amult \ku )\\
\end{zed}
%
From definition and the above immediately follows:
\begin{zed}
  \ran ( (\quaternions \cross \quaternions) \dres (\_ \aplus \_) ) \subset \quaternions\\
  \ran ( (\quaternions \cross \quaternions) \dres (\_ \amult \_) ) \subset \quaternions\\
  \ran (\quaternions \dres (\aneg ))  \subset \quaternions\\
  \ran (\quaternions \dres (\_ \ainv ))  \subset \quaternions\\
\end{zed}
%
Obviously, the real numbers are a subset of the quaternions:
\begin{zed}
  \real = \{~q : \quaternions | \exists q_0 : \real @ q = q_0 \amult
  \aone~\}
\end{zed}
% 
The quaternions form a (four-dimensional) vector space over $\real$:
\begin{zed}
  \forall r1, r2 : \real; q : \quaternions @\\
  \t1 (r1 \aplus r2) \amult q = r1 \amult q \aplus r2 \amult q
\end{zed}
%
and
%
\begin{zed}
  \forall r : \real; q1, q2 : \quaternions @\\
  \t1 r \amult (q1 \aplus q2) = r \amult q1 \aplus r \amult q2 \\
\end{zed}
%
Introducing \emph{conjugate} function:
%%postop \aconj
\begin{axdef}
  \_ \aconj : \quaternions \fun \quaternions
  \where
  \forall q : \quaternions @ \\
  \t1 \exists_1 q_0, q_1, q_2, q_3 :  \real @ \\
  \t2 q = q_0 \amult \aone  \aplus q_1 \amult \iu \aplus q_2
  \amult \ju \aplus q_3 \amult \ku \land\\
  \t1 q\aconj = q_0 \amult \aone \aminus q_1 \amult \iu \aminus q_2 \amult \ju \aminus q_3 \amult \ku \\
\end{axdef}
%
Some laws for the conjugate of quaternions:
\begin{zed}
  \forall q, q1, q2 : \quaternions @ \\
  \t1 (q\aconj)\aconj = q \land \\
  \t1 (q1 \aplus q2)\aconj = q1\aconj \aplus q2\aconj \land \\
  \t1 (q1 \amult q2)\aconj = q2\aconj \amult q1\aconj
\end{zed}
%

The \emph{unit quaternions} are defined as follows:
\begin{zed}
  UnitH == \{~q : \quaternions | q \amult q\aconj = \aone~\}
\end{zed}
%
Another important set is the set of pure quaternions defined as
follows:
\begin{zed}
  PureH == \{~q : \quaternions | \exists_1 q_1, q_2, q_3 : \real @ q =
  q_1 \amult \iu \aplus q_2 \amult \ju \aplus q_3 \amult \ku~\}
\end{zed}

% For interpretation of quaternions as rotations one needs to embed
% $\real3d$ into $\quaternions$:
% %
% \begin{zed}
%   % embdR3H == (\lambda r : \realthreed | (\exists r_1, r_2, r_3 : \real
%   % @ r = (r_1, r_2, r_3)) @ r_1 \amult \iu) 
%   embdR3H == (\lambda r : \realthreed @ first~(first~r) \amult \iu
%   \aplus second~(first~r) \amult \ju \aplus second~r \amult \ku) 
% \end{zed}
%
% \begin{axdef}
%   \quaternions : \power \arithmos
%   \where
%   \complex \subset \quaternions \\
%   \ran ( (\quaternions \cross \quaternions) \dres (\_ \aplus \_) ) \subset \quaternions\\
%   \ran ( (\quaternions \cross \quaternions) \dres (\_ \amult \_) ) \subset \quaternions\\
%   \ran (\quaternions \dres (\aneg ))  \subset \quaternions\\
%   \ran (\quaternions \dres (\_ \ainv ))  \subset \quaternions\\
% \end{axdef}
% Since the complex numbers are included into the quaterions, $\iu$ is
% also member of $\quaternions$. There are two additional special
% constants:

Read \cite{Mukundan2002,Vince2021}
%
\section{Complex Numbers $\complex$}
\label{sec:complex-numbers}
The set of complex numbers $\complex$ is a subset of $\arithmos$,
which contains the real numbers defined in section
\ref{sec:real-numbers}.
%
\begin{axdef}
  \complex : \power \arithmos\\
  \where
  \real \subset \complex \\
  \ran ( (\complex \cross \complex) \dres (\_ \aplus \_) ) \subset \complex\\
  \ran ( (\complex \cross \complex) \dres (\_ \amult \_) ) \subset \complex\\
  \ran (\complex \dres (\aneg ))  \subset \complex\\
  \ran (\complex \dres (\_ \ainv ))  \subset \complex\\
\end{axdef}

The complex numbers are closed with respect to addition,
multiplication, negation and inversion.

There is a special complex constant $\iu$:
%
\begin{axdef}
  \iu : \complex
  \where
  \iu \amult \iu = \aneg \aone
\end{axdef}

Complex numbers have a real and an imaginary part:
%
\begin{axdef}
  \Re, \Im : \complex \fun \real
  \where
  \Re(\azero) = \Im(\azero) = \azero \\
  \Re(\aone) = \aone \land \Im(\aone) = \azero \\
  \Re(\iu) = \azero \land \Im(\iu) = \aone \\
  \forall z_1, z_2: \complex @ ( \Re(z_1) = \Re(z_2) \land \Im(z_1) =
  \Im(z_2) ) \iff z_1 = z_2 \\
  \forall z_1, z_2: \complex @ \Re(z_1 \aplus z_2 ) = \Re(z_1) \aplus
  \Re(z_2) \\
  \forall z_1, z_2: \complex @ \Im(z_1 \aplus z_2 ) = \Im(z_1) \aplus
  \Im(z_2) \\
  \forall z_1, z_2: \complex @ \Re(z_1 \amult z_2 ) = \Re(z_1) \amult
  \Re(z_2) \aminus  \Im(z_1) \amult \Im(z_2) \\
  \forall z_1, z_2: \complex @ \Im(z_1 \amult z_2 ) = \Re(z_1) \amult
  \Im(z_2) \aplus  \Im(z_1) \amult \Re(z_2) \\
  \real = \dom (\Im \rres \{~\azero~\}) \\
\end{axdef}
%
From these rules one can deduce
\begin{zed}
  \forall z : \complex @ z = \Re(z) \aplus \iu \amult \Im(z)
\end{zed}
%
In $\complex$ there is a \emph{conjugate}:
% \begin{axdef}
%   \aconj : \complex \fun \comples
%   \where
  
% \end{axdef}
%%postop \aconj
\begin{axdef}
  \_ \aconj : \complex \fun \complex
  \where
  \forall z: \complex @ z\aconj = \Re(z) \aminus \iu \amult \Im(z) 
%  \aconj == (\lambda z: \complex @ \Re(z) \aminus \iu \amult \Im(z)) 
\end{axdef}
%
From this definition it follows that
\begin{zed}
  % \real = \{~z : \complex @ \aconj~z = z~\}
  \forall z : \complex @ z \in \real \iff z\aconj = z \\
  \forall z: \complex @  (z\aconj)\aconj = z \\
  \forall z_1, z_2 : \complex @ (z_1 \aplus z_2 )\aconj = z_1\aconj \aplus z_2\aconj\\
  \forall z_1, z_2 : \complex @ (z_1 \aminus z_2 )\aconj = z_1\aconj
  \aminus z_2\aconj\\
  \forall z_1, z_2 : \complex @ (z_1 \amult z_2)\aconj = z_1\aconj \amult z_2\aconj\\
\end{zed}
%

\section{Integers and Rational Numbers}
\label{sec:integers-and-rationals}
Unfortunately it is not possible to include the integer numbers into
$\arithmos$, one can only embed them:

\begin{axdef}
  numembd : \num \inj \rat \\
  \where
  numembd~0 = \azero \\
  \forall n : \nat @ numembd (succ~n) = (numembd~n) \aplus \aone \\
  \forall z : \num | z < 0 @ numembd~z = \aneg(numembd(-z)) \\
\end{axdef}

The following laws can be easily derived from the definition:
\begin{zed}
  numembd~1 = \aone \\
  numembd~2 = \atwo \\
  numembd~10 = \aten \\
  \forall i, j : \num @ numembd (i+j) = numembd~i \aplus numembd~j \\
  \forall i, j : \num @ numembd (i-j) = numembd~i \aminus numembd~j \\
  \forall i, j : \num @ numembd (i*j) = numembd~i \amult numembd~j \\
  \forall i, j : \num | i < j @ numembd~i \alt numembd~j \\
\end{zed}

Mixing integers with real and rational numbers in arithmetic
operations would require to use the embedding which would produce
unweildy and clumsy expressions. Therefore, shortcuts for mixed
operations are introduced.
%
%%inop \azplus 3
%%inop \aplusz 3
%%inop \azminus 3
%%inop \aminusz 3
%%inop \azmult 4
%%inop \amultz 4
%%inop \azdiv 4
%%inop \adivz 4
\begin{axdef}
  \_ \azplus \_ : \num \cross \arithmos \fun \arithmos\\
  \_ \aplusz \_ : \arithmos \cross \num \fun \arithmos\\
  \_ \azmult \_ : \num \cross \arithmos \fun \arithmos\\
  \_ \amultz \_ : \arithmos \cross \num \fun \arithmos\\
  \_ \azminus \_ : \num \cross \arithmos \fun \arithmos\\
  \_ \aminusz \_ : \arithmos \cross \num \fun \arithmos\\
  \_ \azdiv \_ : \num \cross \arithmos_1 \fun \arithmos\\
  \_ \adivz \_ : \arithmos \cross (\num \setminus \{~0~\}) \fun \arithmos\\
  \where
  \forall z: \num; x : \arithmos @ z \azplus x = numembd(z) \aplus x =
  x \aplus numembd(z) = x \aplusz z\\
  \forall z: \num; x : \arithmos @ z \azmult x = numembd(z) \amult x =
  x \amult numembd(z) = x \amultz z\\
  \forall z: \num; x : \arithmos @ z \azminus x = numembd(z) \aminus x =
  x \aminus numembd(z) = x \aminusz z\\
  \forall z: \num ; x :\arithmos_1 @ z \azdiv x = numembd(z) \adiv x\\
    \forall z: \num ; x :\arithmos | z \neq 0 @ x \adivz z = x \adiv numembd(z)\\
\end{axdef}

These operations allow a simple formulation of equations such as the
following:

\begin{zed}
 5 \azmult \atwo = \aten \amultz 1 = \aten
\end{zed}
%
Similarily, we need to mix integers and rationals or reals in
relations. Especially there is a need to define equality.

%%inrel \aaeqz
%%inrel \azeqa
%%inrel \aaltz
%%inrel \azlta
%%inrel \azleqa
%%inrel \aaleqz
%%inrel \aagtz
%%inrel \azgta
%%inrel \aageqz
%%inrel \azgeqa
\begin{zed}
  (\_ \aaeqz \_) == \{~x:\real;z:\num | z \azminus x = \azero @ x
  \mapsto z~\}\\
  (\_ \azeqa \_) == \{~x:\real;z:\num | x \aminusz z = \azero @ z
  \mapsto x~\}\\
  (\_ \aaltz \_) == \{~x:\real;z:\num | z \azminus x \agt \azero @ x
  \mapsto z~\}\\
    (\_ \azlta \_) == \{~x:\real;z:\num | x \aminusz z \agt \azero @ z
  \mapsto x~\}\\
  (\_ \azleqa \_) == (\_ \azeqa \_) \cup (\_ \azlta \_)\\
  (\_ \aaleqz \_) == (\_ \aaeqz \_) \cup (\_ \aaltz \_)\\
  (\_ \aagtz \_) == \{~x:\real;z:\num | x \aminusz z \agt \azero @ x
  \mapsto z~\}\\
  (\_ \azgta \_) == \{~x:\real;z:\num | z \azminus x \agt \azero @ z
  \mapsto x~\}\\
  (\_ \aageqz \_) == (\_ \aaeqz \_) \cup (\_ \aagtz \_) \\
  (\_ \azgeqa \_) == (\_ \azeqa \_) \cup (\_ \azgta \_) \\
\end{zed}
Note that the definitions that look identical are of different type,
either subset of $\num \cross \real$ or $\real \cross \num$.

With these definitions it is possible to formulate equations such as
the following in a natural way:
\begin{zed}
  5 \azlta \aten \aaeqz 10
\end{zed}

\section{Additional Functions}
\label{sec:additional-functions}

% \subsection{Power}
% \label{sec:power}

% %%inop \apwr 4
% \begin{axdef}
%   \_ \apwr \_ : \arithmos \cross \num \pfun \arithmos \\
%   \where
%   \forall n: \nat @ \azero \apwr n = \azero\\
%   \forall x: \arithmos @ x \apwr 0 = \aone\\
%   \forall x: \arithmos_1; z : \num @ x \apwr (z+1) = x \amult (x \apwr z)\\
% \end{axdef}

% \subsection{Square Root}
% \label{sec:square-root}
% The square root is a partial function on $\arithmos$.
% \begin{axdef}
%   sqrt : \arithmos \pfun \arithmos
%   \where
%   \forall x : \dom sqrt @ sqrt(x) \amult sqrt(x) = x
% \end{axdef}

% The square root function has the following properties, derived from
% its definition:
% \begin{zed}
%   sqrt(\azero) = \azero\\
%   sqrt(\aone) = \aone\\
%   \realplus \cup \{~\azero~\} \subseteq \dom sqrt\\
%   \dom (sqrt \rres \real) = \realplus \cup \{~\azero~\}\\
%   \ran (\real \dres sqrt) = \realplus \cup \{~\azero~\}\\
% \end{zed}

\subsection{Absolute Value}
\label{sec:abs}

\begin{axdef}
  abs: \arithmos \pfun \realplus \cup \{~\azero~\}\\
  \where
  abs(\azero) = \azero\\
  abs(\aone) = \aone\\
  \forall x, y : \dom abs @ abs(x \amult y) = abs(x) \amult abs(y)\\
  \forall x, y : \dom abs @ abs(x \aplus y) \aleq abs(x) \aplus
  abs(y)\\
  \real \dres abs = \id \real \oplus (\lambda x: \realminus @ \aneg
  x)\\
  \complex \dres abs = (\lambda z: \complex @ sqrt(\Re(z) \apwr 2
  \aplus \Im(z) \apwr 2))\\
  \forall z: \complex @ (abs(z)) \apwr 2 = z \amult z\aconj\\
\end{axdef}

\subsection{Finite Distributed Sum}
\label{sec:fin-distr-sum}
The finite distributed sum of a finite indexed set of elements of
$\arithmos$ is defined as follows:

\begin{gendef}[I]
    \findistsum : (I \ffun \arithmos) \fun \arithmos
    \where
    \findistsum~\emptyset = \azero \\
    \forall f: I \ffun \arithmos | f \neq \emptyset @ \\
    \t1 (\exists i: \dom f @ \findistsum~f = (f~i) \aplus
    \findistsum(\{~i~\} \ndres f))
\end{gendef}
The definition is valid irrespective of the index $i$ due to
commutativity and associativity of the addition in $\arithmos$. 

\subsection{Finite Distributed Product}
\label{sec:fin-distr-product}
Similar to the finite distributed sum there is a finite distributed
product of a finite indexed set of elements of
$\arithmos$ is defined as follows:

\begin{gendef}[I]
    \findistprod : (I \ffun \arithmos) \fun \arithmos
    \where
    \findistprod~\emptyset = \aone \\
    \forall f: I \ffun \arithmos | f \neq \emptyset @ \\
    \t1 (\exists i: \dom f @ \findistprod~f = (f~i) \amult
    \findistprod(\{~i~\} \ndres f))
\end{gendef}
The definition is valid irrespective of the index $i$ due to
commutativity and associativity of the multiplication in $\arithmos$.

\subsection{Maximum of finite Set of real Numbers}
\label{sec:max-real}
For every non-empty finite set of real numbers there is a maximum:
\begin{axdef}
  \realmax : \finset_1 \real \fun \real
  \where
  \forall S: \finset_1 \real @ ( \realmax(S) \in S ) \land (\forall x : S @ x \aleq \realmax(S))
\end{axdef}
For infinite sets the maximum may not exist, but for finite sets it is
unique and the function is well-defined.

\subsection{Minimum of finite Set of real Numbers}
\label{sec:min-real}
There is a similar definition for the minimum of a finite set of real
numbers: 
\begin{axdef}
  \realmin : \finset_1 \real \fun \real
  \where
  \forall S: \finset_1 \real @ ( \realmin(S) \in S ) \land (\forall x : S @ x \ageq \realmin(S))
\end{axdef}

\section{\LaTeX{} Commands}
\label{sec:latex-commands}
This section summarizes the \LaTeX{} commands for typesetting the
number sets defined in this paper (Table~\ref{tab:latex-number-sets}),
the numerical constants (Table \ref{tab:latex-constants}), operations
(Table \ref{tab:latex-unary-operators}) and
\ref{tab:latex-binary-operators}), relations (Table
\ref{tab:latex-relations}) and special functions (Table
\ref{tab:latex-special-functions}).

The \LaTeX{} commands listed here are defined in the style file
\verb|zarithm.sty|.

\begin{table}[htbp]
  \centering
  \begin{tabular}{llp{1cm}llp{1cm}ll}
    $\arithmos$    & \verb|\arithmos| && $\real$ & \verb|\real|&& $\rat$ & \verb|\rat| \\
    $\arithmos_1$  & \verb|\arithmos_1| && $\real_1$ & \verb|\real_1| && $\rat_1$ & \verb|\rat_1| \\
    $\complex$ & \verb|\complex| &&  $\realplus$ & \verb|\realplus| && $\ratplus$ & \verb|\ratplus| \\ 
    $\complex_1$ & \verb|\complex_1| && $\realminus$ &
                                                       \verb|\realminus| && $\ratminus$ & \verb|\ratminus| \\
    $\quaternions$ & \verb|\quaternions| &&  & && &    \\
  \end{tabular}
  \caption{Commands for Number Sets}
  \label{tab:latex-number-sets}
\end{table}
%
\begin{table}[htbp]
  \centering
  \begin{tabular}{llp{1cm}llp{1cm}ll}
    $0$    & \verb|\azero| && $2.0$  & \verb|\atwo| && $\iu$  & \verb|\iu| \\
    $1.0$  & \verb|\aone| && $10.0$ & \verb|\aten| && & \\
  \end{tabular}
  \caption{Commands for $\arithmos$ Constants}
  \label{tab:latex-constants}
\end{table}
%
\begin{table}[htbp]
  \centering
  \begin{tabular}{ll}
    $x\ainv$    & \verb|x\ainv| \\
    $\aneg x$    & \verb|\aneg x| \\
  \end{tabular}
  \caption{Commands for unary Operators}
  \label{tab:latex-unary-operators}
\end{table}
%
\begin{table}[htbp]
  \centering
  \begin{tabular}{llll}
    Operator & Left & Right & Command \\
    \hline \\
    $\aplus$ & $\arithmos, \complex, \real, \rat$ & $\arithmos,
                                                    \complex, \real, \rat$ & \verb|\aplus| \\
    $\aplusz$ & $\num$ & $\arithmos, \complex, \real, \rat$ & \verb|\azplus| \\
    $\azplus$ & $\arithmos, \complex, \real, \rat$ & $\num$ & \verb|\aplusz| \\
    $\aminus$ & $\arithmos, \complex, \real, \rat$ & $\arithmos, \complex, \real, \rat$ & \verb|\aminus| \\
    $\aminusz$ & $\num$ & $\arithmos, \complex, \real, \rat$ & \verb|\azminus| \\
    $\azminus$ & $\arithmos, \complex, \real, \rat$ & $\num$ & \verb|\aminusz| \\
    $\amult$ & $\arithmos, \complex, \real, \rat$ & $\arithmos, \complex, \real, \rat$ & \verb|\amult| \\
    $\amultz$ & $\num$ & $\arithmos, \complex, \real, \rat$ & \verb|\azmult| \\
    $\azmult$ & $\arithmos, \complex, \real, \rat$ & $\num$ & \verb|\amultz| \\
    $\adiv$ & $\arithmos, \complex, \real, \rat$ & $\arithmos_1, \complex_1, \real_1, \rat_1$ & \verb|\adiv| \\
    $\adivz$ & $\num$ & $\arithmos_1, \complex_1, \real_1, \rat_1$ & \verb|\azdiv| \\
    $\azdiv$ & $\arithmos, \complex, \real, \rat$ & $\num \setminus \{0\}$ & \verb|\adivz| \\
    \hline
  \end{tabular}
  \caption{Commands for binary Operators}
  \label{tab:latex-binary-operators}
\end{table}
%
\begin{table}[htbp]
  \centering
  \begin{tabular}{llll}
    Relation & Left & Right & Command \\
    \hline \\
    $=$ & $\arithmos, \complex, \real, \rat$ & $\arithmos, \complex, \real, \rat$ & \verb|=| \\
    $\aaeqz$ & $\arithmos, \complex, \real, \rat$ & $\num$ & \verb|\aaeqz| \\
    $\azeqa$ & $\num$ & $\arithmos, \complex, \real, \rat$ & \verb|\azeqa| \\
    $\alt$ & $\real, \rat$ & $\real, \rat$ & \verb|\alt| \\
    $\aaltz$ & $\real, \rat$ & $\num$ & \verb|\aaltz| \\
    $\azlta$ & $\num$ & $\real, \rat$ & \verb|\azlta| \\
    $\aleq$ & $\real, \rat$ & $\real, \rat$ & \verb|\aleq| \\
    $\aaleqz$ & $\real, \rat$ & $\num$ & \verb|\aaleqz| \\
    $\azleqa$ & $\num$ & $\real, \rat$ & \verb|\azleqa| \\
    $\agt$ & $\real, \rat$ & $\real, \rat$ & \verb|\agt| \\
    $\aagtz$ & $\real, \rat$ & $\num$ & \verb|\aagtz| \\
    $\azgta$ & $\num$ & $\real, \rat$ & \verb|\azgta| \\
    $\ageq$ & $\real, \rat$ & $\real, \rat$ & \verb|\ageq| \\
    $\aageqz$ & $\real, \rat$ & $\num$ & \verb|\aageqz| \\
    $\azgeqa$ & $\num$ & $\real, \rat$ & \verb|\azgeqa| \\
    \hline
  \end{tabular}
  \caption{Commands for Relations}
  \label{tab:latex-relations}
\end{table}
%
\begin{table}[htbp]
  \centering
  \begin{tabular}{ll}
    $abs(x)$    & \verb|abs(x)| \\
    $x \apwr n$    & \verb|x \apwr n| \\
    $sqrt(x)$    & \verb|sqrt(x)| \\
    $\aconj(z)$    & \verb|\aconj(z)|\\
    Finite distributes sum & \verb|\findistsum| \\
    Finite distributes product & \verb|\findistprod| \\
    $\realmax$ & \verb|\realmax| \\
    $\realmin$ & \verb|\realmin| \\
  \end{tabular}
  \caption{Commands for special Functions}
  \label{tab:latex-special-functions}
\end{table}
% 
% bibliography
\begin{thebibliography}{99}
\bibitem{Valentine2012}
	Sam Valentine,
  	\textit{Definitions of Numbers in Z},
    2012
  \bibitem{Spivey1998}
    J. M. Spivey,
    \textit{The Z Notation: A Reference Manual},
    Prentice Hall International (UK) Ltd,
    1998,
    2nd edition
  \bibitem{ISO13568}
    ISO/IEC 13568:2002(E),
    \textit{Information technology -- Z formal specification notation -- Syntax, type system and semantics},
    2002
  \bibitem{Vince2021}
    John Vince,
    \textit{Quaternions for Computer Graphics},
    Springer London,
    2021
  \bibitem{Mukundan2002}
      Ramakrishnan Mukundan,
      \textit{Quaternions: From classical mechanics to computer graphics, and beyond},
      Proceedings of the 7th Asian Technology Conference in Mathematics,
      2002
\end{thebibliography}

\end{document}
%%% Local Variables:
%%% mode: latex
%%% TeX-master: t
%%% End:
