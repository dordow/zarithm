\documentclass[12pt]{scrartcl}

\usepackage[T1]{fontenc}
\usepackage[english]{babel}
\usepackage[backend=biber]{biblatex}
\addbibresource{/home/frank/docs/reference/literature.bib}
\AtEveryBibitem{\clearfield{url}}

\usepackage{graphicx}
\usepackage{url}
\usepackage{tikz}
\usepackage{units}
\usepackage{gensymb}
\usepackage{amsthm}
\usepackage{zarithm}

\usepackage[mark]{gitinfo2}
\usepackage[outerbars]{changebar}
\usepackage[pdf,shell]{dottex}

%\usepackage{hyperref} % should be last but does not work then with zed-csp

\usepackage{savesym}
\usepackage{zed-csp}
\savesymbol{because}
%\usepackage[nothing]{todo}
\usepackage{todo}
\restoresymbol{TXF}{because}


%\nochangebars

\newcommand{\remark}[1]{ \vspace{1em}\textbf{\underline{#1}}\vspace{1em} }
%\newcommand{\comment}[1]{ \textbf{\underline{#1}} }

% \newcommand{\arithmos}{\mathbb{A}}
% \newcommand{\real}{\mathbb{R}}
% \newcommand{\rat}{\mathbb{Q}}

% \newcommand{\azero}{0}
% \newcommand{\aone}{1}
% \newcommand{\aplus}{+}
% \newcommand{\amult}{*}
% \newcommand{\aminus}{-}
% \newcommand{\adiv}{/}
% \newcommand{\aneg}{-}
% \newcommand{\ainv}{^{-1}}
% \newcommand{\alt}{<}
% \newcommand{\aleq}{\leq}
% \newcommand{\agt}{>}
% \newcommand{\ageq}{\geq}

%\newcommand{\Arithmos}{\mathbb{A}}
\newcommand{\Real}{\mathbb{R}}
\newcommand{\Rational}{\mathbb{Q}}

\newcommand{\azero}{0}
\newcommand{\aone}{1}
\newcommand{\aplus}{+}
\newcommand{\amult}{*}
\newcommand{\aminus}{-}
\newcommand{\adiv}{/}
\newcommand{\aneg}{-}
\newcommand{\ainv}{^{-1}}
\newcommand{\alt}{<}
\newcommand{\aleq}{\leq}
\newcommand{\agt}{>}
\newcommand{\ageq}{\geq}

%%% Local Variables:
%%% mode: latex
%%% TeX-master: t
%%% End:


\newtheorem{prop}{Proposition}[section]
\newtheorem{schemadef}{Schema}[section]
\newtheorem{genericdef}{Generic Definition}[section]
\newtheorem{zdef}{Definition}[section]

\begin{document}

%opening
\title{Arithmetic with Z and Fuzz}

\author{Frank Dordowsky}

% make the title area
\maketitle

\begin{abstract}
Goal is to create a prelude for fuzz that copes with arithmetics
according to \cite{Valentine2012} or \cite{Arthan1996}.
\end{abstract}

\tableofcontents
%\newpage

\section{Introduction}
\label{sec:intro}

\section{Arithmos $\arithmos$}
\label{sec:arithmos}

\begin{zdef}[Arithmos]
  \label{zdef:arithmos}
  $\arithmos$, which can be pronounced “arithmos”, is the encompassing
  set of numbers whose whole extent is deliberately left
  unspecified. This definition has been adopted from
  \cite{Valentine2012}. 
  \begin{zed}
    [\arithmos] 
  \end{zed}
\end{zdef}

%%inop \aplus 3
%%inop \aminus 3
\begin{zdef}[Laws of Addition]
  \label{zdef:laws-addition}
  \begin{axdef}
    \_ \aplus \_ : \arithmos \cross \arithmos \fun \arithmos\\
    \_ \aminus \_ : \arithmos \cross \arithmos \fun \arithmos\\
    \aneg : \arithmos \fun \arithmos\\
    \azero : \arithmos\\
    \aone : \arithmos\\
    \where
    \forall x : \arithmos @ x \aplus \azero = \azero \aplus x = x \\
    \forall x : \arithmos @ x \aplus (\aneg x) = \azero \\
    \forall x : \arithmos @ \aneg (\aneg x) = x \\
    \forall x,y: \arithmos @ x \aplus y = y \aplus x \\
    \forall x,y,z: \arithmos @ ( x \aplus y ) \aplus z = x \aplus (y
    \aplus z) \\
    \forall x,y : \arithmos @ x \aminus y = x \aplus (\aneg y)\\
  \end{axdef}
\end{zdef}

\begin{zdef}[Non-zero Numbers]
  \label{zdef:non-negative-numbers}
  \begin{zed}
    \arithmos_1 == \arithmos \setminus \{ \azero \}
  \end{zed}
\end{zdef}


%%inop \amult 4
%%inop \adiv 4
%%postop \ainv
\begin{zdef}[Laws of Multiplication]
  \label{zdef:laws-multiplication}
  \begin{axdef}
    \_ \amult \_ : \arithmos \cross \arithmos \fun \arithmos\\
    \_ \adiv \_ : \arithmos \cross \arithmos_1 \fun \arithmos\\
    \_ \ainv : \arithmos_1 \fun \arithmos_1\\
    \where
    \forall x : \arithmos @ x \amult \aone = \aone \amult x = x \\
    \forall x : \arithmos_1 @ x \amult (x \ainv) = \aone \\
    \forall x : \arithmos_1 @ (x \ainv)\ainv = x \\
    \forall x,y: \arithmos @ x \amult y = y \amult x \\
    \forall x,y,z: \arithmos @ ( x \amult y ) \amult z = x \amult (y
    \amult z) \\
    \forall x: \arithmos; y : \arithmos_1 @ x \adiv y = x \amult (y
    \ainv)\\
    \forall x: \arithmos @ x \amult \azero = \azero \amult x =
    \azero\\
    \forall x, y, z: \arithmos @ x \amult ( y \aplus z ) = x \amult y
    \aplus x \amult z \\
  \end{axdef}
\end{zdef}


\section{Real Numbers $\real$}
\label{sec:real-numbers}

\begin{zdef}[Real Numbers]
  \label{zdef:real-numbers}
  \begin{axdef}
    \real : \power \arithmos\\
    \where
    % \azero \in \real\\
    % \aone \in \real\\
    \ran ( (\real \cross \real) \dres (\_ \aplus \_) ) \subset \real\\
    \ran ( (\real \cross \real) \dres (\_ \amult \_) ) \subset \real\\
    \ran (\real \dres (\aneg ))  \subset \real\\
    \ran (\real \dres (\_ \ainv ))  \subset \real\\
  \end{axdef}
\end{zdef}

%%inrel \alt
\begin{zdef}[Total Order on Real Numbers]
  \label{zdef:order-on-reals}
  \begin{axdef}
    \_ \alt \_ : \real \rel \real\\
    \where
    \forall x : \real @ \lnot x \alt x\\
    \forall x, y: \real @ x \alt y \lor x = y \lor y \alt x\\
    \forall x, y, z: \real | x \alt y \land y \alt z @ x \alt z \\
    \azero \alt \aone\\
  \end{axdef}
\end{zdef}

%%inrel \aleq
%%inrel \agt
%%inrel \ageq
\begin{zed}
  (\_ \aleq \_) == (\_ \alt \_ ) \cup \id \real \\
  (\_ \agt \_) == (\_ \alt \_ ) \inv \\
  (\_ \ageq \_) == (\_ \agt \_ ) \cup \id \real \\
\end{zed}

\section{Rational Numbers $\rat$}
\label{sec:rational-numbers}


\begin{axdef}
  \rat : \power \real
  \where
  \azero \in \rat\\
  \aone \in \rat\\
  \ran ( (\rat \cross \rat) \dres (\_ \aplus \_) ) \subset \rat\\
  \ran ( (\rat \cross \rat) \dres (\_ \amult \_) ) \subset \rat\\
  \ran (\rat \dres (\aArithmetic in Z.neg ))  \subset \rat\\
  \ran (\rat \dres (\_ \ainv ))  \subset \rat\\
\end{axdef}

\section{Integers and Rational Numbers}
\label{sec:integers-and-rationals}

\begin{axdef}
  embd : \num \inj \rat \\
  \where
  embd~0 = \azero \\
  embd~1 = \aone \\
  \forall i, j : \num @ embd (i+j) = embd~i \aplus embd~j \\
  \forall i, j : \num @ embd (i*j) = embd~i \amult embd~j \\
  \forall i, j : \num | i <j @ embd~i \alt embd~j \\
\end{axdef}


% This section contains some generic definitions that are not part of
% the mathematical toolsset of Z.

% \begin{zdef}
%   \label{zdef:abs}
%   \begin{axdef}
%     abs: \num \fun \num \where \forall z: \num @ abs(z) = \IF z \geq 0
%     \THEN z \ELSE -z
%   \end{axdef}
% \end{zdef}

% \begin{genericdef}[ID]
%   \label{genericdef:id}
%   \begin{gendef}[X]
%     id: X \fun X \where id = \{~x: X @ x \mapsto x~\}
%   \end{gendef}
% \end{genericdef}

% \begin{zdef}[Time]
%   \label{zdef:time}
%   \begin{zed}
%     TIME == \nat \\
%   \end{zed}
% \end{zdef}

% %%unchecked
% \begin{zed}
%   TimeSteps \dres tmStp = id
% \also
%   \forall t: TIME @ tmStp(t) \leq t < tmStp(t) + 100
% \end{zed}


% \begin{prop}
%   \label{prop:some-proposition}
%   \begin{argue}
%     \vdash \forall ic: \num; sf: TimeSteps \fun VOLT; ts: TimeSteps | \\
%     \t1   ts \ne period @ \\
%     \t2 int(ic,sf)(ts+100) - int(ic,sf)(ts) = 100*ic*sf(ts)
%   \end{argue}
% \end{prop}

% \begin{proof}
%   The proof is as follows:

%   \begin{argue}
%     \forall ic: \num; sf: TimeSteps \fun VOLT; ts: TimeSteps | ts \ne
%     period \also ts + 100 \in TimeSteps \and ts + 100 \ne 0 \also
%     int(ic,sf)(ts+100) = int(ic,sf)(ts) + 100*ic*sf(ts)
%   \end{argue}
%   The last line follows from the definition of the integrator applied to
%   values of time steps, and the lemma follows from the last line above.
% \end{proof}

% We formalize the specification of the calibration algorithm (which is
% performed manually) with the following Z schema:

% %%unchecked
% \begin{schema}{Calibration}
%   Bcal?: BW \\
%   Tcal?: TEMP \\
%   mc?: ModuleCharacteristics
% \also
%   ro!: RampOffsets \\
%   rs!: RampSets \\
%   ic!: \num \\
%   fm!: FREQ
% \where
%   \forall TS: SensorTempSteps; ts: TimeSteps @\\ 
% \t1    rs!(Bcal?,TS,ts) = rs!(Bcal?,snsTempMin,ts)
% \also
%   \forall t: TIME @ \\
% \t1    abs(freqRampOf(ic!,ro!,rs!,mc?)(Bcal?,Tcal?,t) -\\
% \t2      idealRamp(Bcal?,fm!)(t)) \leq eps
% \end{schema}


% \begin{argue}
%   [DetermineIntegrationConstant] \vdash \forall ts: TimeStep | ts \ne period @ \\
% \t1   Ucal?(ts+100) - Ucal?(ts) = \\
% \t2     int(ic!,rampSlopeOf(rs?)(Bcal?,Tcal?))(ts+100) - \\
% \t3       int(ic!,rampSlopeOf(rs?)(Bcal?,Tcal?))(ts) = \\
% \t1     100*ic!*rampSlopeOf(rs?)(Bcal?,Tcal?)(ts) =    \\
% \t1   100*ic!*dac(rs?(Bcal?,snsTmpStp(Tcal?),ts))
% \end{argue}

% 
% bibliography
\printbibliography{}

\todos

\end{document}
%%% Local Variables:
%%% mode: latex
%%% TeX-master: t
%%% End:

%%% Local Variables:
%%% mode: latex
%%% TeX-master: t
%%% End:
