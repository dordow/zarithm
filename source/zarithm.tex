\documentclass[12pt]{article}

\usepackage[T1]{fontenc}
\usepackage[english]{babel}

\usepackage{zed-csp} % must be included after zarithm
\usepackage{zarithm}
\usepackage[mark]{gitinfo2}
\newcommand{\remark}[1]{ \vspace{1em}\textbf{\underline{#1}}\vspace{1em} }
%\newcommand{\comment}[1]{ \textbf{\underline{#1}} }

\begin{document}

%opening
\title{Arithmetic Definitions in  Z to be used with Fuzz}

\author{Frank Dordowsky}

% make the title area
\maketitle

\begin{abstract}
The intention of this \LaTeX\ file is to extend the mathematical
tool-kit of the Z reference manual to include rational and real
numbers. 
\end{abstract}

%\tableofcontents
%\newpage

\section{Introduction}
\label{sec:intro}
Rational and real numbers are often used in the specification of
embedded systems to represent environmental variables such as
distances, termperatures, voltage levels etc. The mathematical toolkit
of the Z reference manual (ZRM \cite{Spivey1998}) does not contain
rational and real numbers, whereas the number toolkit of the ISO Z
standard does. However, moving to standard (ISO) Z \cite{ISO13568} is
not always the favourite option, especially if a large amount of Z
specifications in ZRM style is available. Moreover, most Z textbooks
have been written prior to ISO Z. For those who wish to stick with
ZRM, this set of two files (\texttt{zarithm.sty} and
\texttt{zarithm.tex}, this file) are intended to provide a solution.

The file \texttt{zarithm.sty} is only used for typesetting, whereas
\texttt{zarithm.tex} includes the Z definitions of the types for
rational and real numbers, their basic operations and numerical
comparisons The Z definitions are sufficient for type setting, not for
automated proof. If the user intends to use autoated proof, there is
probably no better way than to transit to standard Z.

The ZRM comes along with the tool \texttt{fuzz}, a typechecker. The
approach described here makes use of the fact that fuzz accepts more
than one \LaTeX file and remembers the Z definitions of the files
provided first.

\section{Arithmos $\arithmos$}
\label{sec:arithmos}

$\arithmos$, which can be pronounced “arithmos”, is the encompassing
set of numbers whose whole extent is deliberately left
unspecified. This definition has been adopted from
\cite{Valentine2012}.

\begin{zed}
  [\arithmos] 
\end{zed}

$\arithmos$ is encoded (in \LaTeX) by "\texttt{\textbackslash arithmos}".

$\arithmos$ is a field with respect to two operations $\aplus$ and
$\amult$, and neutral elements $\azero$ and $\aone$. Note that these
are not identical to the elements $0$ and $1$ defined in $\num$.

The addition is encoded by "\texttt{\textbackslash{}aplus}". There is
a unary operator $\aneg$ to negate any number, encoded as
"\texttt{\textbackslash{}aneg}"
%%inop \aplus 3
%%inop \aminus 3
\begin{axdef}
  \_ \aplus \_ : \arithmos \cross \arithmos \fun \arithmos\\
  \_ \aminus \_ : \arithmos \cross \arithmos \fun \arithmos\\
  \aneg : \arithmos \fun \arithmos\\
  \azero : \arithmos\\
  \aone : \arithmos\\
  \atwo : \arithmos\\
  \aten : \arithmos\\
  \where
  \forall x : \arithmos @ x \aplus \azero = \azero \aplus x = x \\
  \forall x : \arithmos @ x \aplus (\aneg x) = \azero \\
  \forall x : \arithmos @ \aneg (\aneg x) = x \\
  \forall x,y: \arithmos @ x \aplus y = y \aplus x \\
  \forall x,y,z: \arithmos @ ( x \aplus y ) \aplus z = x \aplus (y
  \aplus z) \\
  \forall x,y : \arithmos @ x \aminus y = x \aplus (\aneg y)\\
\end{axdef}

\begin{zed}
  \arithmos_1 == \arithmos \setminus \{ \azero \}
\end{zed}

%%inop \amult 4
%%inop \adiv 4
%%postop \ainv
\begin{axdef}
  \_ \amult \_ : \arithmos \cross \arithmos \fun \arithmos\\
  \_ \adiv \_ : \arithmos \cross \arithmos_1 \fun \arithmos\\
  \_ \ainv : \arithmos_1 \fun \arithmos_1\\
  \where
  \forall x : \arithmos @ x \amult \aone = \aone \amult x = x \\
  \forall x : \arithmos_1 @ x \amult (x \ainv) = \aone \\
  \forall x : \arithmos_1 @ (x \ainv)\ainv = x \\
  \forall x,y: \arithmos @ x \amult y = y \amult x \\
  \forall x,y,z: \arithmos @ ( x \amult y ) \amult z = x \amult (y
  \amult z) \\
  \forall x: \arithmos; y : \arithmos_1 @ x \adiv y = x \amult (y
  \ainv)\\
  \forall x: \arithmos @ x \amult \azero = \azero \amult x =
  \azero\\
  \forall x, y, z: \arithmos @ x \amult ( y \aplus z ) = x \amult y
  \aplus x \amult z \\
\end{axdef}

The encoding of the constants is shown in table
\ref{tab:arithmos-constant-encoding}. 

\begin{table}[htbp]
  \centering
  \begin{tabular}{ll}
    $0$    & \texttt{\textbackslash{}azero} \\
    $1.0$  & \texttt{\textbackslash{}aone} \\
    $2.0$  & \texttt{\textbackslash{}atwo} \\
    $10.0$ & \texttt{\textbackslash{}aten} \\
  \end{tabular}
  \caption{Coding $\arithmos$ Constants}
  \label{tab:arithmos-constant-encoding}
\end{table}
%
\section{Real Numbers $\real$}
\label{sec:real-numbers}
Real numbers are a subset $\real$ of Arithmos, which contains the
constants shown in Table \ref{tab:arithmos-constant-encoding}.

\begin{axdef}
  \real : \power \arithmos\\
  \where
  \azero \in \real\\
  \aone \in \real\\
  \atwo \in \real\\
  \aten \in \real\\
  \ran ( (\real \cross \real) \dres (\_ \aplus \_) ) \subset \real\\
  \ran ( (\real \cross \real) \dres (\_ \amult \_) ) \subset \real\\
  \ran (\real \dres (\aneg ))  \subset \real\\
  \ran (\real \dres (\_ \ainv ))  \subset \real\\
\end{axdef}

The real numbers are abgeschlossen with respect to addition,
multiplication, negation and inversion.

For real numbers, there is a relation ``$<$'', encoded as
\texttt{\textbackslash{}alt}:

%%inrel \alt
\begin{axdef}
  \_ \alt \_ : \real \rel \real\\
  \where
  \forall x : \real @ \lnot x \alt x\\
  \forall x, y: \real @ x \alt y \lor x = y \lor y \alt x\\
  \forall x, y, z: \real | x \alt y \land y \alt z @ x \alt z \\
  \azero \alt \aone\\
\end{axdef}

The definition of the other relations $\aleq$, $\agt$ and $\ageq$ can
be traced back to this relation:

%%inrel \aleq
%%inrel \agt
%%inrel \ageq
\begin{zed}
  (\_ \aleq \_) == (\_ \alt \_ ) \cup \id \real \\
  (\_ \agt \_) == (\_ \alt \_ ) \inv \\
  (\_ \ageq \_) == (\_ \agt \_ ) \cup \id \real \\
\end{zed}
They are encoded as \texttt{\textbackslash{}aleq},
\texttt{\textbackslash{}agt} and \texttt{\textbackslash{}ageq}.


There are shortcuts for positive and negative real numbers, encoded as
\texttt{\textbackslash{}realplus} and
\texttt{\textbackslash{}realminus}.

\begin{zed}
  \realplus == \{~r :\real | r \agt \azero~\}\\
  \realminus == \real \setminus ( \realplus \cup \{~\azero~\})
\end{zed}

\section{Rational Numbers $\rat$}
\label{sec:rational-numbers}

\begin{axdef}
  \rat : \power \real
  \where
  \azero \in \rat\\
  \aone \in \rat\\
  \atwo \in \rat\\
  \aten \in \rat\\
  \ran ( (\rat \cross \rat) \dres (\_ \aplus \_) ) \subset \rat\\
  \ran ( (\rat \cross \rat) \dres (\_ \amult \_) ) \subset \rat\\
  \ran (\rat \dres (\aneg ))  \subset \rat\\
  \ran (\rat \dres (\_ \ainv ))  \subset \rat\\
\end{axdef}

\begin{zed}
  \ratplus == \{~q :\rat | q \agt \azero~\}\\
  \ratminus == \rat \setminus ( \ratplus \cup \{~\azero~\})
\end{zed}

\section{Integers and Rational Numbers}
\label{sec:integers-and-rationals}
Unfortunately it is not possible to include the integer numbers into
$\arithmos$, one can only embed them:

\begin{axdef}
  numembd : \num \inj \rat \\
  \where
  numembd~0 = \azero \\
  \forall n : \nat @ numembd (succ~n) = (numembd~n) \aplus \aone \\
  \forall z : \num | z < 0 @ numembd~z = \aneg(numembd(-z)) \\
\end{axdef}

The following laws can be easily derived from the definition:
\begin{zed}
  numembd~1 = \aone \\
  numembd~2 = \atwo \\
  numembd~10 = \aten \\
  \forall i, j : \num @ numembd (i+j) = numembd~i \aplus numembd~j \\
  \forall i, j : \num @ numembd (i-j) = numembd~i \aminus numembd~j \\
  \forall i, j : \num @ numembd (i*j) = numembd~i \amult numembd~j \\
  \forall i, j : \num | i < j @ numembd~i \alt numembd~j \\
\end{zed}

Mixing integers with real and rational numbers in arithmetic
operations would require to use the embedding above which would look
clumsy. Therefore, shortcuts for mixed operations are introduced.

%%inop \azplus 3
%%inop \aplusz 3
%%inop \azminus 3
%%inop \aminusz 3
%%inop \azmult 4
%%inop \amultz 4
%%inop \azdiv 4
%%inop \adivz 4
\begin{axdef}
  \_ \azplus \_ : \num \cross \arithmos \fun \arithmos\\
  \_ \aplusz \_ : \arithmos \cross \num \fun \arithmos\\
  \_ \azmult \_ : \num \cross \arithmos \fun \arithmos\\
  \_ \amultz \_ : \arithmos \cross \num \fun \arithmos\\
  \_ \azminus \_ : \num \cross \arithmos \fun \arithmos\\
  \_ \aminusz \_ : \arithmos \cross \num \fun \arithmos\\
  \_ \azdiv \_ : \num \cross \arithmos_1 \fun \arithmos\\
  \_ \adivz \_ : \arithmos \cross (\num \setminus \{~0~\}) \fun \arithmos\\
  \where
  \forall z: \num; x : \arithmos @ z \azplus x = numembd(z) \aplus x =
  x \aplus numembd(z) = x \aplusz z\\
  \forall z: \num; x : \arithmos @ z \azmult x = numembd(z) \amult x =
  x \amult numembd(z) = x \amultz z\\
  \forall z: \num; x : \arithmos @ z \azminus x = numembd(z) \aminus x =
  x \aminus numembd(z) = x \aminusz z\\
  \forall z: \num ; x :\arithmos_1 @ z \azdiv x = numembd(z) \adiv x\\
    \forall z: \num ; x :\arithmos | z \neq 0 @ x \adivz z = x \adiv numembd(z)\\
\end{axdef}

Similarily, we need to mix integers and rationals or reals in
relations. Especially there is a need to equality.

%%inrel \aaeqz
%%inrel \azeqa
%%inrel \aaltz
%%inrel \azlta
%%inrel \azleqa
%%inrel \aaleqz
%%inrel \aagtz
%%inrel \azgta
%%inrel \aageqz
%%inrel \azgeqa
\begin{zed}
  (\_ \aaeqz \_) == \{~x:\arithmos;z:\num | z \azminus x = \azero @ x
  \mapsto z~\}\\
  (\_ \azeqa \_) == \{~x:\arithmos;z:\num | x \aminusz z = \azero @ z
  \mapsto x~\}\\
  (\_ \aaltz \_) == \{~x:\arithmos;z:\num | z \azminus x \agt \azero @ x
  \mapsto z~\}\\
    (\_ \azlta \_) == \{~x:\arithmos;z:\num | x \aminusz z \agt \azero @ z
  \mapsto x~\}\\
  (\_ \azleqa \_) == (\_ \azeqa \_) \cup (\_ \azlta \_)\\
  (\_ \aaleqz \_) == (\_ \aaeqz \_) \cup (\_ \aaltz \_)\\
  (\_ \aagtz \_) == \{~x:\arithmos;z:\num | x \aminusz z \agt \azero @ x
  \mapsto z~\}\\
  (\_ \azgta \_) == \{~x:\arithmos;z:\num | z \azminus x \agt \azero @ z
  \mapsto x~\}\\
  (\_ \aageqz \_) == (\_ \aaeqz \_) \cup (\_ \aagtz \_) \\
  (\_ \azgeqa \_) == (\_ \azeqa \_) \cup (\_ \azgta \_) \\
\end{zed}

\section{Special Functions}
\label{sec:special-functions}

\subsection{Absolute Value}
\label{sec:abs}

\begin{axdef}
  abs: \arithmos \pfun \realplus \cup \{~\azero~\}\\
  \where
  abs(\azero) = \azero\\
  abs(\aone) = \aone\\
  \forall x, y : \dom abs @ abs(x \amult y) = abs(x) \amult abs(y)\\
  \forall x, y : \dom abs @ abs(x \aplus y) \aleq abs(x) \aplus
  abs(y)\\
  \real \dres abs = \id \real \oplus (\lambda x: \realminus @ \aneg x)\\
\end{axdef}

\subsection{Power}
\label{sec:power}

%%inop \apwr 4
\begin{axdef}
  \_ \apwr \_ : \arithmos \cross \num \pfun \arithmos \\
  \where
  \forall n: \nat @ \azero \apwr n = \azero\\
  \forall x: \arithmos @ x \apwr 0 = \aone\\
  \forall x: \arithmos; z : \num @ x \apwr (z+1) = x \amult (x \apwr z)\\
\end{axdef}

\subsection{Square Root}
\label{sec:square-root}
The square root is a partial function on $\arithmos$.
\begin{axdef}
  sqrt : \arithmos \pfun \arithmos
  \where
  \forall x : \dom sqrt @ sqrt(x) \amult sqrt(x) = x
\end{axdef}

The square root function has the following properties, derived from
its definition:
\begin{zed}
  sqrt(\azero) = \azero\\
  sqrt(\aone) = \aone\\
  \realplus \cup \{~\azero~\} \subseteq \dom sqrt\\
  \dom (sqrt \rres \real) = \realplus \cup \{~\azero~\}\\
  \ran (\real \dres sqrt) = \realplus \cup \{~\azero~\}\\
\end{zed}

\subsection{Finite Distributed Sum}
\label{sec:fin-distr-sum}
The finite distributed sum of a finite indexed set of elements of
$\arithmos$ is defined as follows:

\begin{gendef}[I]
    \findistsum : (I \ffun \arithmos) \fun \arithmos
    \where
    \findistsum~\emptyset = \azero \\
    \forall f: I \ffun \arithmos | f \neq \emptyset @ \\
    \t1 (\exists i: \dom f @ \findistsum~f = (f~i) \aplus
    \findistsum(\{~i~\} \ndres f))
\end{gendef}
The definition is valid irrespective of the index $i$ due to
commutativity and associativity of the addition in $\arithmos$. 

\subsection{Finite Distributed Product}
\label{sec:fin-distr-product}
Similar to the finite distributed sum there is a finite distributed
product of a finite indexed set of elements of
$\arithmos$ is defined as follows:

\begin{gendef}[I]
    \findistprod : (I \ffun \arithmos) \fun \arithmos
    \where
    \findistprod~\emptyset = \aone \\
    \forall f: I \ffun \arithmos | f \neq \emptyset @ \\
    \t1 (\exists i: \dom f @ \findistprod~f = (f~i) \amult
    \findistprod(\{~i~\} \ndres f))
\end{gendef}
The definition is valid irrespective of the index $i$ due to
commutativity and associativity of the multiplication in $\arithmos$.

\subsection{Maximum of finite Set of real Numbers}
\label{sec:max-real}
For every non-empty finite set of real numbers there is a maximum:
\begin{axdef}
  \realmax : \finset_1 \real \fun \real
  \where
  \forall S: \finset_1 \real @ ( \realmax(S) \in S ) \land (\forall x : S @ x \aleq \realmax(S))
\end{axdef}
For infinite sets the maximum may not exist, but for finite sets it is
unique and the function is well-defined.

\subsection{Minimum of finite Set of real Numbers}
\label{sec:min-real}
There is a similar definition for the minimum of a finite set of real
numbers: 
\begin{axdef}
  \realmin : \finset_1 \real \fun \real
  \where
  \forall S: \finset_1 \real @ ( \realmin(S) \in S ) \land (\forall x : S @ x \ageq \realmin(S))
\end{axdef}


% 
% bibliography
\begin{thebibliography}{99}
\bibitem{Valentine2012}
	Sam Valentine,
  	\textit{Definitions of Numbers in Z},
    2012
  \bibitem{Spivey1998}
    J. M. Spivey,
    \textit{The Z Notation: A Reference Manual},
    Prentice Hall International (UK) Ltd,
    1998,
    2nd edition
  \bibitem{ISO13568}
    ISO/IEC 13568:2002(E),
    \textit{Information technology -- Z formal specification notation -- Syntax, type system and semantics},
    2002
\end{thebibliography}

\end{document}
%%% Local Variables:
%%% mode: latex
%%% TeX-master: t
%%% End:
